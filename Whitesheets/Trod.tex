\documentclass[white]{gl2018}
\usepackage{scrextend}
\begin{document}
\name{\wTrod{}}

\section*{The Trod}
\newcommand{\areastart}[1]{ \begin{huge}{\bf \#{#1}} \end{huge} \\}
\newenvironment{sect}[1]{\begin{minipage}{\textwidth}\areastart{#1}\\}{\end{minipage}\vspace{3in}}

Follow the instructions in the player system doc, duplicated here for your convenience:

\subsection*{Using the Trod}

Game space contains one trod (a doorway between the mortal world and the hedge). To use the trod (in either direction), go to the two white incense burners in front of the garden in the middle of the asphalt outside the dining hall. If you want to, and are able, walk up the steps from the asphalt, between the two burners. Stop, turn around, and pass back between the burners as you descend the stairs to the asphalt. If you can’t, or don’t want, to climb the stairs, just declare your intent to use the Trod. Check your watch and note the time, then proceed as follows.

Find the first condition that matches the time you passed between the incense burners, turn to that number, and do what it says. {\em Do not look at any other items.}
\begin{enumerate}
\item If the time is before 10 am on Saturday, Oct 27th, including any time on Friday, go to \#1.
\item 
If the time includes at least 2 zeros (i.e.: 1\textbf{0:0}4, 3:\textbf{00}), go to \#2.
\item If the hour includes 2 digits, (i.e.: \textbf{10}:32, \textbf{12}:14 but not \textbf{2}:40), go to \#3.
\item 
If the time includes at least one 4 (i.e.: \textbf{4}:\textbf{4}8, 9:\textbf{4}2), go to \#4.
\item 
If the last digit in the time is divisible by three (i.e: 1:0\textbf{3}, 3:1\textbf{9}), go to \#5.
\item
If the ones digit of the hour is even (i.e: \textbf{2}:01, \textbf{8}:30), go to \#6.
\item
If the last digit is 5, (i.e: 1:5\textbf{5}, 7:2\textbf{5}), go to \#7. 
\item If the tens digit of the minutes is odd (i.e.: 9:3\textbf{7}, 5:1\textbf{1}), go to \#8.
\item
If the last digit is prime (i.e: 7:2\textbf{7}, 5:0\textbf{2}; note that neither 0 nor 1 are prime), go to \#9.
\item If the ones digit of the hour is composite (i.e: \textbf{9}:20, \textbf{6}:28;  0, 1 and 2 are NOT composite), go to \#10.
\item If the time includes at least one 1 (i.e.: {\bf 1}:08, 5:2{\bf 1}), go to \#11.
\item If the ones digit of the hour is odd (i.e: {\bf 3}:28, {\bf 7}:20), go to \#12.
\end{enumerate}
\pagebreak
\begin{sect}{1}
You successfully transition either from the mortal world to the hedge, or the hedge to the mortal world, depending which you started in.

If you are now in the hedge, take a head band and put it on, or carry it with you in another obvious manner if you have concerns with the tactile stimulation, so there is a visual cue for people to know that you are in the hedge.

If you are returning to the mortal world from the hedge, return the headband to the bin for others to use.

You can only interact with people who are also in the same realm as you. You cannot see or hear into the other realm, or affect it in any way unless you have an ability that lets you do so, or a specific mechanic describes a way for you to do so. Do not act on information you may encounter from the other realm (i.e.: an overheard conversation, knowledge of the fact that someone is in the other realm)
\end{sect}
\newcommand{\jumpto}[1]{You step through the portal and find yourself at #1{}. You are still in the hedge and have not transitioned back to the mortal world.

\textit{Use the OOC symbol of fist on your head or in front of your forehead to go directly to #1{} (which can be found IRL at #1{\MYwhere}), then go back in character. You may not interact with anyone along the way; tell anyone who asks who was watching at the trod that you vanish just like normal.}}
\begin{sect}{2}
\jumpto{\pAlbinoRedwoodWeekend}
\end{sect}
\begin{sect}{3}
\jumpto{\pReflectingPoolWeekend}
\end{sect}
\begin{sect}{4}
\jumpto{\pOutInTheForestWeekend}
\end{sect}
\begin{sect}{5}
You step through the portal and find yourself at the Sculpture Garden.  You are still in the hedge and have not transitioned back to the mortal world.

\textit{Use the OOC symbol of fist on your head or in front of your forehead to go directly to Go to the playground area and look for a box labeled ``\sSculptureGardenWarning{}''.  This mechanic grants you access to bypass the sign and interact with the stuff inside.  From there, return back here to return to the mortal world.  You may not interact with anyone along the way; tell anyone who asks who was watching at the trod that you vanish just like normal.}}
\end{sect}
\begin{sect}{6}
\jumpto{\pOldLectureHallWeekend}
\end{sect}
\begin{sect}{7}

You step through the portal and find yourself in a Sculpture Garden (go to “Area G” and bypass the sign that says “Stop, do not enter unless you know otherwise.”). You are still in the hedge and have not transitioned back to the mortal world. Any lit candles you and your group have blow out in an errant gust of wind.

Use the OOC symbol of fist on your head or in front of your forehead to go directly to Area G, then go back in character. You may not interact with anyone along the way; tell anyone who asks at the trod that you vanish just like normal.
\end{sect}
\begin{sect}{8}
\jumpto{\pOldLectureHallWeekend}
\end{sect}
\begin{sect}{9}
You step through the portal and find yourself at the Sculpture Garden.  You are still in the hedge and have not transitioned back to the mortal world.

\textit{Use the OOC symbol of fist on your head or in front of your forehead to go directly to Go to the playground area and look for a box labeled ``\sSculptureGardenWarning{}''.  This mechanic grants you access to bypass the sign and interact with the stuff inside.  From there, return back here to return to the mortal world.  You may not interact with anyone along the way; tell anyone who asks who was watching at the trod that you vanish just like normal.}}
\end{sect}
\begin{sect}{10}
\jumpto{\pOutInTheForestWeekend}
\end{sect}
\begin{sect}{11}
\jumpto{\pReflectingPoolWeekend}
\end{sect}
\begin{sect}{12}
\jumpto{\pAlbinoRedwoodWeekend}
\end{sect}
\end{document}
