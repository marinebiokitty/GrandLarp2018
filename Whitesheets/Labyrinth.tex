\documentclass[white]{gl2018}
\begin{document}
\name{\wLabyrinth{}}
\section*{The Hedge Maze}
\newcommand{\areastart}[1]{ \begin{huge}{\bf \#{#1}} \end{huge} \\}
\newenvironment{fromhere}{\begin{center}\begin{itshape}}{\end{itshape}\end{center}}
% \newcommand{\fromhere}[1]{\begin{fromhere} #1 \end{fromhere}}
\newenvironment{location}[1]{\begin{minipage}{\textwidth}\areastart{#1}\\}{\end{minipage}\vspace{1in}}
\section*{Instructions}
The hedge maze is represented by a ``choose your own adventure'' style booklet. 

\begin{itemize}
\item Each entry will contain a description of what you see, and instructions on what actions are available at the location and/or what directions you can go from that point.  
\item In some entries, you may have to make a check or suffer some (minor) penalty, or make a check in order to go a certain direction. As usual, you may always assist in checks unless it is specified that each person must make their own. 
\item Only the navigation options listed at each point are available to you; you may not backtrack unless an entry specifically tells you that you may. Some entries will allow you to leave the maze without``fleeing'' (See below).
\item You may enter the maze alone or with a party of any number. At any point that there is a decision to be made, the party may split. You may join up with other groups at the same location (same number) as you. 
\item {\bf Flight}: You may escape the maze at any time by throwing caution to the winds and {\em fleeing}. If you do so, you must take: 1 point of physical damage to represent minor injuries during your escape, or add 1 point to your $\Psi$ (Psi)Score to represent the more malevolent mental effects of the maze getting to you. If you find your narrative to be significantly impacted by these consequences in ways that don't facilitate your play, the GMs encourage you to roleplay appropriate mental and emotional distress for a few minutes or longer after escaping the maze. If you flee the maze, you must start over at the entrance if you wish to return to the maze. 
\item The beginning of the maze is at \#1.
\end{itemize}
\pagebreak
\begin{large}
\begin{location}{1}
You stand at the entrance to the maze, looking up at the 10 foot tall walls made of some dark foliage. It never has flowers, even in the spring, but it always has thorns, and occasionally small berries. Even at the best of times, under the noon-day sun, the maze feels cold and creepy. Long shadows drift across the paths, and out of the corners of your eyes, you sometimes think you see movement. Now the very air feels sinister and malevolent. 

\begin{fromhere}If you wish to enter the maze, go to \#2.\end{fromhere}
\end{location}
\begin{location}{2}
The path forks a dozen steps into the maze. The two paths look identical.  
\begin{fromhere}
If you go left, go to \#3. \\
If you go right, go to \#29.\\
You may leave the maze at this time without suffering the consequences of fleeing.
\end{fromhere}
\end{location}
\begin{location}{3}
You reach an intersection. Your options are: a dark narrow path that is somewhat overgrown, a wide, bright, well manicured path, or a path that appears...normal.  Just normal.

\begin{fromhere}
 If you go take the dark path, go to \#26.\\
 If you take the wide path, go to \#35.\\
 If you take the normal path, go to \#2.
\end{fromhere}
\end{location}
\begin{location}{4}
You shake yourself, rubbing your eyes and willing life into your muscles.  The forest has gone still and silent.  There is a hint of something cloyingly sweet tinging the air.  You hurry along the path, though it is a struggle to force your legs to move.  Each breath is more and more of a struggle, but something tells you this is not a safe place to stop and rest.  Finally,  you come around a bend in the path and are greeted with a burst of birdsong and a gust of wind.  The fatigue drops from your body and you feel well enough to carry on.  
\begin{fromhere}
Proceed to \#24.
\end{fromhere}
\end{location}
\begin{location}{5}
% teddy bear
The feather on the teddy bear's scales is the Feather of Truth: any statement made in its presence is weighed against the feather.  

\textit{Any statement spoken in the teddy bear's presence causes the scales to shift, reflecting whether or not the statement was true or false to the best of the speaker's knowledge.  All present can see whether the speaker is telling the truth or lying.}

You may stay as long as you like, speaking with each other and seeing the teddy bear's judgement. 

\begin{fromhere}
You may proceed whenever you like to \#40.  The hedge closes behind you.
\end{fromhere}
\end{location}
\begin{location}{6}
You step into the mirror as though it were no more than a doorway. For a moment you are completely disoriented. Up is down, and right is left, and the sun definitely goes around the earth, and then you are stepping out of a fountain dry as a bone. Your candle is still doused though.

\begin{fromhere}
Go to \#38.\\

\end{fromhere}
\end{location}
\begin{location}{7}
You travel down the path. After turning left, you see the path forks: one straight, one path right. You hear running water from the fork to the right. 
\begin{fromhere}
If you continue straight, go to \#17.\\
If you turn right,  go to \#38.\\
  If you retrace your steps, go to \#29.
\end{fromhere}
\end{location}
\begin{location}{8}
You notice the path becomes less rocky as you go, with roots and ivy increasingly strewn across the path.  Just as you begin to take conscious notice, the ivy vines suddenly animate and whip around your feet, working their way rapidly up your legs.  You sense that there is something important in this direction, and that the plants have been placed as a barrier.  The vines seem to be dragging you deeper into the trees, which seem dark and foreboding.  Make a physical check at 11.
\begin{fromhere}
If you succeed, go to \#34.\\
  If you fail, go to \#41.
\end{fromhere}
\end{location}
\begin{location}{9}
You begin to retrace your steps, only to meet turns you definitely don't remember taking. You eventually reach a T intersection with a path that looks a lot more normal, but not necessarily familiar.
\begin{fromhere}
 If you go left, go to \#20.\\
 If you go right, go to \#17.
\end{fromhere}
\end{location}
\begin{location}{10}
You walk down the gently sloping path.  A breeze stirs the leaves in the trees and you can hear the chatter of robins and chickadees.  Everything seems, for a moment, refreshingly ordinary, and you feel yourself start to relax.  Suddenly, you start as out of the corner of your eye you see a human-sized fox standing on its hind legs leaning against a tree.  It nods to you and flashes a toothy grin.  Its teeth look very sharp.  You shiver and hurry down the path, glancing behind you all the way.  It takes all of your willpower not to run.  The fox does not appear to follow you - but would you really see if it did?

You come to an intersection of four paths.  Your options are: a cobblestone path, a dirt track that looks little wider than a goat path, a wide open path, and a low staircase.  

\begin{fromhere}
If you choose the cobblestone path, go to \#24.\\
  If you choose the goat path, go to \#32.\\
  If you choose the wide path, go to \#35.\\
  If you choose the staircase, go to \#23.
\end{fromhere}
\end{location}
\begin{location}{11}
You are lightly gored before you manage to flee in panic.  The minotaur gives chase, and you can feel the heat of his breath behind you.  Finally you manage to squeeze through a hole in the hedges and escape.  You take two damage and increase your $\Psi$ Psi Score by one, and exit the maze, visibly hurt and smelling of sweat and blood.  
\begin{fromhere}
You may, if you choose, begin again at \#1, or you may exit the maze.
\end{fromhere}
\end{location}
\begin{location}{12}
You find a bed of moonwort ferns.  You may pick some, if you like.  Keep track of how many ferns you have.

\begin{fromhere}
You may return to \#15, or escape the maze without penalty.
\end{fromhere} 
\end{location}
\begin{location}{13}
You reach a T-intersection, and can go in any direction.
\begin{fromhere}
  If you take the first path, go to \#15.\\
 If you take the second path, go to \#33.\\
 If you take the third path, go to \#30.
\end{fromhere}
\end{location}
\begin{location}{14}
You do whatever it takes to stay calm and steady.  Eventually, you turn a corner and the urgency fades. 
\begin{fromhere}Continue to \#10.\end{fromhere}\end{location}
\begin{location}{15}
You continue on your way and soon reach a clearing with a small fountain and a three way intersection.  There is a small, brightly colored dragon, barely longer than your forearm, bathing in the fountain!  When it sees you approach, it lets out a small jet of flame and flies through a wall in the maze.  You're not sure whether it was an illusion, the wall is an illusion, or what just happened.

\begin{fromhere}
   If you go left, go to \#12.\\
 If you go right, go to \#32.\\ 
If you go straight, go to \#13.\\
If you attempt to chase the dragon through the wall of the maze, go to \#28.
\end{fromhere}
\end{location}
\begin{location}{16}
You successfully engage the minotaur in combat. After trading a few good blows, you land a solid one, and the beast howls in pain and retreats. You continue on your way, a bit shaken, to another clearing. The clearing is a dead end, but has a huge, beautiful mirror erected in the middle. The mirror is slightly concave. Engraved around the sides of the more than full length mirror are the words ``Seeing is Believing. Be careful what you see.''\\

\emph{This mirror allows you to scry on far away locations, and for particularly skilled users, to look into their own past, or the past of someone else who stands at the focal point of the mirror. There is no mechanic for this mirror, but feel free to incorporate it into your story if it serves.}\\

This path is a dead end, but you aren't sure you wish to risk the ire of a wounded Minotaur. Your options are to \emph{flee}, or step into the mirror. What could go wrong?

\begin{fromhere}
   If you step into the mirror, go to \#6.\\
\end{fromhere}

\end{location}
\begin{location}{17}
The path starts to slope downwards, gradually at first, then steeper.  As you pick your way down, you hear a strange bird call in the trees ahead of you.  It echoes eerily through the trees and almost sounds like a child crying.  The hair on the back of your neck rises but you continue down the path.  A moment later, a bird calls close behind you with a cry like a child’s scream and a loud beating of wings.  Make a physical check at 11.  If you fail the check, you lose your footing and tumble the rest of the way down the path, taking one point of physical damage.  You realize you cannot climb back up and face a three-way intersection.  
\begin{fromhere}
If you go left, go to \#40.\\
  If you go right, go to \#20.\\
  If you go straight, go to \#7.\\
\end{fromhere}
\end{location}
\begin{location}{18}
You continue to press on, but more and more you have to force your way through. Make a physical check at 11. If you fail the check, one person in your group must take one point of physical damage. 
\begin{fromhere}Go to \#30.\end{fromhere}
\end{location}
\begin{location}{19}
As you travel this path, you hear a peculiar noise. It sounds like the quintessential plant growing sound from the movies. On a hunch, you look behind you - just in time to watch the path behind you vanish behind a new wall of the maze that grows in the space of a few heartbeats to the full 10 feet. If you hadn't seen it with your own eyes, you wouldn't believe it. The new wall looks every bit as mature as the ones next to it. If you aren't used to strange, inexplicable things happening, this unsettles you.  Each member of your party must increase their $\Psi$ (Psi) score by 1 if their Wyrd is less than 3.
\begin{fromhere}  Continue to \#37.\end{fromhere}
\end{location}
\begin{location}{20}
As you continue down the path, you start to feel tired. How long have you been walking? The path widens out a bit, and this seems like a good place to take rest.
\begin{fromhere} If you stop and rest, go to \#36.\\
 If you press on, go to \#4.
\end{fromhere}
\end{location}
\begin{location}{21}
If you have reached this point, you are in possession of the star-metal key, and know that it opens Brendan Cloche's diary. The diary, however, is not here. Someone has found it and taken it. Check the $\langle$OOC Channel$\rangle$ on Discord and look for the most recent message that says ``I have the Journal.'' The key will now guide you to that person.  Roleplay accordingly. 
\begin{fromhere}Go to \#24.\end{fromhere}
\end{location}
\begin{location}{22}
You encounter a wide clearing with a centerpiece: a beautiful but visibly menacing bush of black roses.  Each rose is almost enclosed within a tangle of long, sharp thorns, which move slightly at your approach.  As you watch, one petal falls off a rose -- and disintegrates at almost the first touch of a thorn.\\

To get a petal from one of these roses will require being {\em very} careful.\\

You may either make a Physical check as a group at difficulty 16, or each person may make a Wyrd check of 7 or lower.  Characters may choose not to participate.
\begin{itemize}
\item If the party succeeds on the Physical check, one party member may take a black rose petal and consume it immediately. 
\item If the party fails on the Physical check, one participating party member must take one magical damage.
\item If one or more party members succeed on the Wyrd check, one party member (who may or may not have participated or succeeded) may consume a black rose petal immediately.
\item If nobody succeeded on the Wyrd check, one participant must take one magical damage.
\end{itemize}
The effects of eating a petal are as follows:
\begin{itemize}
\item If you have already eaten a petal today, there is no effect.
\item If your Wyrd is equal to 1, 2, or 7, your $\Psi$ (Psi) score is reset to zero.
\item If your Wyrd is not equal to 1, 2, or 7, your Wyrd is reduced by one.
\end{itemize}
You may decide amongst yourselves who will consume the petal.

{\bf Magical Damage:} Magical damage does not heal naturally. You will not regain that health by sleeping. You will need to find some sort of magical healing to recover.
\begin{fromhere}
You may exit the maze without penalty, or return to \#28.
\end{fromhere}
\end{location}
\begin{location}{23}
As you climb the broad, narrow stairs, you wonder a bit. From the outside, there is no sign of a hill inside the maze. But you are definitely going up. The path forks again. The path to the left is clearly sloping down. The path to the right flattens out.
\begin{fromhere} If you turn left, go to \#10.\\ If you turn right,  go to \#19.\end{fromhere}
\end{location}
\begin{location}{24}
Your path winds and turns onto a straight cobblestone path, the cool stones under your feet feeling somehow reassuring.  You have a sense that one direction points into the depths of the maze, and the other direction points...somewhere else, though you aren't precisely certain where.  
\begin{fromhere}To go further into the depths of the maze, go to \#10.\\  To go somewhere else, go to \#42.\end{fromhere}
\end{location}
\begin{location}{25}
You encounter a satyr, sitting on a log and heating water for tea over a campfire. He is softly playing a pipe.

When he notices you, he motions you over.  ``You are from the University, yes?'' he asks.  ``A wonderful place!  Full of magic.  Very good.  I do not wish to see it destroyed...but great danger is coming.  You must stand together to protect it...or before too long, there will be no more students traipsing through this maze to provide me company.''

Despite your questioning, he supposedly knows little more than that danger is coming, and that unity will protect you.  He offers to escort you out of the maze.
\begin{fromhere}
You may exit the maze at no penalty, or return to \#33.
\end{fromhere}
\end{location}
\begin{location}{26}
The path gets more overgrown as it twists and wind on its way, not meeting any other paths. You wonder if this is a path at all, or if you have gotten lost. 
\begin{fromhere}If you turn back, go to \#9.\\ If you press on, go to \#18.\end{fromhere}
\end{location}
\begin{location}{27}
You hurry down the path,  faster and faster. Behind you, you think you hear something. You turn to look over your shoulder to see a horseman, all in black, bearing down on you with a sabre raised high to strike. 

Every party member must make a Wyrd check of 5 or lower.  Darklings may make a check of 7 or lower. 
\begin{fromhere}
If any party member succeeds, go to \#44.\\
 If all party members fail, go to \#31.
\end{fromhere}
\end{location}
\begin{location}{28}
You pass through the wall, disoriented and confused, and feeling some sort of fizz in the air as you pass through.  Gain one $\Psi$ (Psi) point in your disorientation.  Turning around, you find the wall has gone solid behind you: you cannot return.  The dragon is nowhere to be seen.  You find a fork in the path before you: a brick path to the right, and a dirt path to the left.  
\begin{fromhere}To go left, go to \#22.\\  To go right, go to \#40.\end{fromhere}
\end{location}
\begin{location}{29}
You reach an intersection. You can continue straight down a nondescript path, or you can turn left, and start to climb a shallow staircase.  
\begin{fromhere}If you go straight, turn to \#7.\\ If you turn left, go to \#23.\end{fromhere}
\end{location}
\begin{location}{30}
There is a small alcove along the path. It contains a statue of a young satyr, playing his pipes. Is that music you can hear in the distance? 
\begin{fromhere}Go to \#13.\end{fromhere}
\end{location}
\begin{location}{31}
You fight the horseman!  Make a party physical check at 14.  If you fail, distribute two points of damage among the party; otherwise, the horseman is defeated.
\begin{fromhere}
Proceed to \#43.
\end{fromhere}
\end{location}
\begin{location}{32}
You notice, half-buried under leaves and detritus, what seems like a picture: scraping some of it away, you find, embedded in the very stones of the path, a mosaic depicting Piper of the Forest.  It is breathtaking, and you are surprised such beautiful art is somewhere so difficult to find.
\begin{fromhere}
You may proceed along the path in either direction, to \#10 or \#15.
\end{fromhere}
\end{location}
\begin{location}{33}
You reach a fork in the path.  From the right, you hear faint music.
\begin{fromhere}The path on the left leads to \#13.\\
The path on the right leads to \#25.
\end{fromhere}
\end{location}
\begin{location}{34}
You rip free of the vines, and continue on your way. It is almost as if something were trying to stop you or prevent you taking this path. You press on and round a corner. In front of you is an odd clearing. In there is rubble everywhere, as though a stone building once stood here. A few rotting wooden pillars stand - some framework long collapsed. A large pit in the middle of the space is partially filled in with more broken rubble. On close examination you realise it is mostly plaster. This place has a timeless quality you are starting to associate with places on campus that are not normally accessible by mundane means. This is an old bell furnace, constructed in the style popular in Europe after the first World War. It looks large enough that the Cloche Tower bell could have been forged here… Each party member makes a wyrd check of 4 or lower to search the area for anything of interest. If you have a Star-metal Key with you (a specific item created by the GMs), the holder of the key may make the check at 6. Winter and Summer courtiers get an additional +1 bonus to the target. This bonus may stack with the previous one if they both apply. 

\begin{fromhere}
If any party member succeeds the check, go to \#39.\\
 If the entire party fails the check, you find nothing of GM plot relevance. Go to \#24.
\end{fromhere}
\end{location}
\begin{location}{35}
Traveling this path is easy. The ground is flat, and the path is wide. As you stroll down it though you start to feel exposed - and observed. 
\begin{fromhere}If you pick up your pace and start to hurry, go to \#27.\\
 If you continue to walk at the same pace, resisting the urge to hurry, go to \#14.
\end{fromhere}
\end{location}
\begin{location}{36}
You realize as you begin to take a rest that you are under the effects of a sleep spell, but too late: you fall asleep.  You don't know how much time passes before you awaken suddenly, confused and disoriented.  You survey your surroundings: you are somewhere you don't recognize, in a clearing, completely different from where you were when you stopped to rest.  There is a path in front of you, but it seems green and faintly menacing.  There are no other landmarks or pathways in sight.
\begin{fromhere}Go to \#8.\end{fromhere}
\end{location}
\begin{location}{37}
You turn just in time to hear the enraged roar of a minotaur.  He towers above you, eight feet of muscle, sinew, and sharp-looking horns.  You wonder how you could have missed the smell: it is pungent, smelling of sweat and perhaps a hint of blood.  You may choose to run from the minotaur, or fight by making a physical check at 14.
\begin{fromhere}
  If you run away, go to \#33.\\
If you fight and succeed on the check, go to \#16. \\
 If you fight and fail the check, go to \#11.
\end{fromhere}
\end{location}
\begin{location}{38}
The path opens up into a small courtyard. In the middle of the courtyard is a modest fountain.  Bright yellow, orange, and red leaves float in the water, despite there being no trees nearby that could have dropped them. This is a dead end.
\begin{fromhere} You may backtrack to \#7, or leave the maze entirely {\em without} suffering the consequences of fleeing (and begin again at \#1, if you choose).
\end{fromhere}
\end{location}
\begin{location}{39}
\emph{For the party members who succeeded on the check at the previous location:}

 Check in the J packet at the station.  If the packet contains an item, you may examine the item and decide if you wish to take it.  If there is no item, and you do not have the Star-metal Key, or you have the key, but don't know what it is for, you find nothing of GM plot relevance. 
\begin{fromhere}
If you take the item, follow the OOC instructions on the item immediately, then go to \#24.\\
If you do have the star-metal key, and you know the OOC information inside the key, go to \#21.
\\
All other party members should go to the same place.
\end{fromhere}
\end{location}
\begin{location}{40}
As you walk along this straight path, it feels as though you are in a dream.  The light around you seems oddly bright and...sharp, in some way that feels unreal.  You ought to get out of here.
\begin{fromhere}
You may go along this path in either direction, to either \#17 or \#28.
\end{fromhere}
\end{location}
\begin{location}{41}
The vines reach up to your neck now, cutting off circulation.  As you pass out, you hear the sound of running...hoofbeats?  You wake up just outside the maze, smelling something spicy in the air, and as you groggily come to your senses, you see hoofprint tracks: one set, deeper, leading to your body, and one much shallower set leading back into the maze.  Follow the instructions for fleeing the maze.
\end{location}
\begin{location}{42}
A majestic tree towers above you, branches laden with fruit.  It seems like it ought to be visible for miles, but you didn't notice it until you came here, beside its trunk.

The fruit it bears is Amaranthine, a healing goblin fruit.  Party members may each heal one physical damage.  Characters may only benefit from this once per day; it is dangerous to consume too much goblin fruit.
\begin{fromhere}
You may exit the maze at no penalty, or return to \#24.
\end{fromhere}
\end{location}
\begin{location}{43}
You walk along the path and find, propped up against a tree, a small, battered, teddy bear, with a set of scales, weighted on one side with a feather.
\begin{fromhere}
If you speak at all as you pass, go to \#5.\\
Otherwise, go to \#40, finding the hedge closes behind you.
\end{fromhere}
\end{location}
\begin{location}{44}

You realize that the horseman isn't real! It is somehow a trick of the shadows. A very convincing one, but nonetheless, ultimately an illusion. You grit your teeth and stand your ground. The sabre passes harmlessly through you and the horseman vanishes into the wall of the maze. 
\begin{fromhere} Go on to \#15.\end{fromhere}
\end{location}
\end{large}
\end{document}
