%%%%%
%%
%% This file sets up the Mem, MemFold, and MemEnvelope datatypes, and
%% creates possible macros for each.
%%
%% The Mem datatype isn't really used directly; it's there so the
%% other datatypes can inherit and share its code.
%%
%%%%%

\DECLARESUBTYPE{Mem}{Element}
\PRESETS{Mem}{
  %% \MYname is the trigger
  \F\MYtext	%%  text
  }


%%%%%
%% MemFold and MemEnvelope are both subtypes of Mem.  MemFold is for
%% fold-n-staple style mempackets, MemEnvelope is for stuff-n-seal
%% style mempackets.  If you want a mempacket to contain interesting
%% contents, like sheets, abilities, and other mempackets, use a
%% MemEnvelope.
\DECLARESUBTYPE{MemFold}{Mem}
\DECLARESUBTYPE{MemEnvelope}{Mem}


%%%%%
%% MemCover and MemPage are for the cover and pages of mempacket
%% booklets, which resemble research notebooks.  These are good
%% substitutes for large piles of MemFolds, and can be useful for
%% things like amnesiac characters.
%%
%% Like MemFolds, MemPages shouldn't directly own any other elements
%% as contents.  Use MemEnvelope instead.
%%
%% MemPages are usually assigned to a MemCover (via \MYmems), with the
%% MemCover representing the entire booklet and assigned to a
%% character.
%%
%% A MemCover is not a mempacket in and of itself; its name is not its
%% trigger and its text is not a memory.
\DECLARESUBTYPE{MemCover}{Mem}
\PRESETS{MemCover}{
  \sd\MYtext	{Each page is a memory/event packet with a separate
		trigger.}
  }

\DECLARESUBTYPE{MemPage}{Mem}


%%%%%
%% \memfold{<trigger>}{<text>}
%% \memenvelope{<trigger>}{<text>}
%% \memcover{<name>}{<pages>}
%% \mempage{<trigger>}{<text>}
%% \startmembook{<name>} <pages> \endmembook
%%
%% These are wrappers around \INSTANCE, useful as 1-shots.
%% \startmembook...\endmembook is a simple wrapper around \memcover
%% that may have better syntax for use within character sheets (inside
%% a \starttag{mems}...\endtag block).
\newinstance{MemFold}{\memfold[2]}{
  \s\MYname{#1}\s\MYtext{#2}}
\newinstance{MemEnvelope}{\memenvelope[2]}{
  \s\MYname{#1}\s\MYtext{#2}}
\newinstance{MemCover}{\memcover[2]}{
  \s\MYname{#1}\s\MYmems{#2}}
\newinstance{MemPage}{\mempage[2]}{
  \s\MYname{#1}\s\MYtext{#2}}

\long\def\startmembook#1#2\endmembook{\memcover{#1}{#2}}


%%%%%%%%%%%%%%%%%%%%%%%%%%%%%%%%%%%%%%%%%%%%%%%%%%%%%%%%%%%%%%%%%%

\NEW{MemFold}{\mTest}{
  \s\MYname	{Test Trigger}
  \s\MYtext	{This is a Test of a fold-n-staple memory packet}
  }
  
%%%%%%%%%%%%%%%%%%%%%%%%%%%%%%%%%%%%%%%%%%%%%%%%%%%%%%%%%%%%%%%%%%


\NEW{MemEnvelope}{\mPoolVisionOne}{
  \s\MYname	{A Vision of Terror}
  \s\MYtext	{Thirst. Darkness. Abandoned. You are entombed beneath the earth, your entrails rotting. Your throat is parched and bleeding. Dust and dry leaves fill your mouth. Your eyes are sewn shut. Serpents entwine and pierce your flesh, strangling, smothering. Then relief, cool water flowing through dark veins, bursting from your throat like truth. Again the cleansing moonlight purges, again the eyes gaze. And so all shall be its shadow, for only the shadow can cast the light, never again to be silenced and starved, or four shall sunder, and three feast. 
	}
}

\NEW{MemEnvelope}{\mPoolVisionTwo}{
  \s\MYname	{A Vision of Terror}
  \s\MYtext	{A bride stands before you, still as a statue. Swirling mists form her white gown and veil, her face utterly obscured. In her right hand she holds a bell, in her left, a pewter chalice, and on her head rests a wreath of woven evergreen stems. Tears spring unbidden to your eyes, though you know not why. Dark, menacing shapes grope through the fog like tentacles. Years drift by with the mist, till it feels like nearly an entire century has passed. The fog begins to thin and flow down the bride’s frame like melting snow, her features growing more distinct. But as they do, the probing tendrils of darkness grow nearer and more distinct as well. The implements the bride bears are changing too, the bell forming cracks, the chalice tarnishing, the wreath withering before your eyes. There is something eerily familiar about the bride, but before you can pinpoint what, one of the fingers of shadow reaches her and envelopes her in darkness. The bride lets out one bone-chilling shriek and is silenced. 
	}
}
 
\NEW{MemEnvelope}{\mPoolVisionThree}{
  \s\MYname	{A Vision of Terror}
  \s\MYtext	{A woman you recognize from campus statuary to be Piper of the Forest is strolling through her beloved forest of redwoods, her steps dogged by a dense fog. She meanders hither and yon, oblivious to the mist marking her path like a jet contrail. As you watch though, the fog begins to thin, and you can see that she has been leaving a trail of blooming wildflowers with each step. You hear the baying of hunting hounds, and see the monstrous beasts burst from a thornbush, following her path. Piper tries to flee, but the hounds are soon upon her, dragging her down kicking and screaming. But the last wisps of fog reach out like a tentacle and seizes her ankle. A vicious tug-of-war ensues, the venerable founder getting dragged back and forth through thickets of brambles which catch and snare. The struggle is not over when the vision ends, but the hounds appear to be winning.  
	}
}

\NEW{MemEnvelope}{\mPoolVisionFour}{
  \s\MYname	{A Vision of Terror}
  \s\MYtext	{You’re in the old Dramaturge Opera House, watching an opera about the history of the University. Out of the corner of your eye, you keep glimpsing a vague figure with a pale face, each time in a different place in the audience, but every time you look it is gone. The first act is marked by a dense fog which rolls across the stage and into the audience, before gradually thinning. At the end of the first act, when the house lights go up for intermission, you see that some of the audience appears to have vanished. When the curtain rises on the second act, instead of the orchestra’s music, you hear the loud clanging of a belltower. Again you catch glimpses of the shadowy figure here and there in the audience. When the second act ends, the house lights reveal that more people have vanished. A sense of foreboding rises within you as the third and final act begins. Once again you see the phantom in the audience, never in the same place twice. At the end of the third act, instead of thunderous applause, you hear a single pair of hands clapping, loudly and sharply, like the crack of a whip. The house lights go up and the entire audience is gone, except for the phantom seated in the VIP box. You see it clearly now, a dapper figure wearing a white opera mask, top hat, and navy blue tailcoat, clutching a silver cane. The phantom ceases clapping, raises one white-gloved hand, and gives a sweeping bow. All the lights in the theater are snuffed out in an instant, and with them, all hope.
	}
}

\NEW{MemEnvelope}{\mPoolVisionFive}{
  \s\MYname	{A Vision of Terror}
  \s\MYtext	{Before you towers a library of breathtaking scale, stack upon stack of moldering tomes stretching as far as the eye can see. Spidery hands with the wrong number of fingers reach out and pull a fresh book from a shelf. You see the title, THE TALE OF THE WAXEN MASQUE, on the cover, and then the hands open the book to the first page. But the page is completely blank. The spidery hands reach out, across time and space, probing through a rolling fog, and return clutching buildings. With a shock, you recognize them as structures and landmarks from the University, including the belltower, the albino redwood, and the amphitheater. The hands carefully place them onto the blank page, where the buildings transform into 3-dimensional, paper cutouts, like a pop-up book. The hands turn to the second page of the book, which is also blank, and resume their search. You hear the belltower toll flatly from within the book, and the hands come back empty. They turn to the third page (also blank), and this time, drag back what seems like every last bit of the University. The page becomes crammed with paper buildings, the entirety of Wax-Chandler enclosed within the covers of the book. With a satisfied flourish and a sound like thunder, the hands slam shut the covers of the book, and then seal it with a lock. 
	}
}

\NEW{MemEnvelope}{\mPoolVisionSix}{
  \s\MYname	{A Vision of Terror}
  \s\MYtext	{You are watching something akin to a ``Wile E. Coyote'' cartoon, with one key difference: Someone dressed as Wax Chandler University’s mascot is playing the role of the roadrunner. You watch the coyote set one of its elaborate traps, but it loses track of its prey in a dense fog. Nonetheless, in a display of cartoonish grotesquery, the snare still manages to decapitate the mascot. The rest of the body continues to race around jerkily while the coyote mounts the head on a wall. Unsatisfied, the coyote continues to try to catch the rest of its quarry. This time it sets up a contraption to drop a heavy weight on its target, but it misfires and the weight lands on the coyote’s head with a deafening clang that reverberates like a bell. Undeterred, the coyote makes a third attempt, the most elaborate of all, and this time succeeds. The scene ends with the sickening image of a cartoon coyote, its eyes suddenly blazing with a nightmare hunger, tearing into its victim in a feeding frenzy of blood and gore. 
	}
}

\NEW{MemEnvelope}{\mPoolVisionSeven}{
  \s\MYname	{A Vision of Terror}
  \s\MYtext	{A hand mirror lies shattered into three fragments on the ground, two of the shards still clinging to the frame, the third cast far afield. Reflected in the lone shard, you glimpse a host of fleeting images; among them are a ram’s horn, a scroll covered in beautiful calligraphy, a robin, and tufts of fur snagged in a thornbush dripping blood, with flowers of breathtaking beauty blooming from each place a drop lands. In the two shards still joined to the frame, you see a marionette. In one shard, the puppet walks free, jauntily building towers of toy blocks, seemingly oblivious to its own strings, which hang limply. In the other, spidery hands with too many fingers manipulate the strings with an eerie grace, making the marionette dance and knock down the very same towers of blocks it is building in the other shard. The marionette only moves in one of the two shards at a time, its counterpart hanging still and lifeless in the other shard until they switch. Inexplicably, you smell the scent of peaches. 
	}
}

\NEW{MemEnvelope}{\mPoolVisionEight}{
  \s\MYname	{A Vision of Terror}
  \s\MYtext	{ 
	}
}

\NEW{MemEnvelope}{\mPoolVisionNine}{
  \s\MYname	{A Vision of Terror}
  \s\MYtext	{ 
	}
} 
%%%%%%%%%%%%%%%%%%%%%%%%%%%%%%%%%%%%%%%%%%%%%%%%%%%%%%%%%%%%%%%%%%
