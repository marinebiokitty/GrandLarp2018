
%%%%
%%
%% This file sets up the Sign and Label datatypes and creates Sign and
%% Label macros.
%%
%% Signs generally represent interesting parts of game area, usually
%% as things posted on walls.  Labels represent other things, often on
%% or inside envelopes, that are part of complex mechanics.
%%
%% The default value for \MYloc will inherit location from the Place
%% or Sign most immediately up the ownership tree.  Override this by
%% setting \MYloc to anything (even blank).
%%
%% Sign is for full-sized signs that would cover most of a large
%% manila envelope; SignMedium is for signs sized to half-sized manila
%% envelopes; SignSmall is for signs sized for small manila envelopes
%% (the same size as item cards).  Label, LabelMedium, and LabelSmall
%% are analagous, but they don't have a \takedownby note at the
%% bottom.  You can always use a sign or label without an envelope or
%% with a differently-sized envelope.  Choose which based on
%% visibility and content.
%%
%% SignTiny is for signs you want to be hard to find; it is small and
%% does not have a \takedownby note.  SignDot is for a very small
%% "dot" which only has a title.
%%
%% SignStrip produces a strip of paper (without a \takedownby note)
%% with labels on the outside that show on both sides if you fold it
%% in half.  These are a convenient alternative to sub-envelopes. They
%% can also be used for "s-packets" taped to walls (see
%% Extras/README-s-packets).
%%
%% LabelCover produces a label similar to the cover to a research
%% notebook.  LabelPage, likewise, produces a page.
%%
%% EOG is for full-sized end-of-game signs.
%%
%%%%%

\DECLARESUBTYPE{Sign}{Element}
\PRESETS{Sign}{
  \FD\MYloc	{\mylocation} %% real-space location
	\FD\MYtime {\mytime} %% timeframe sign applies
  \FD\MYtext	{} %% text of sign
  }
\POSTSETS{Sign}{
  \edef\mylocation{\MYloc}
  \protected@edef\@ownerstring{%
    \MYname%
    \ifx\mylocation\empty\else\ (\mylocation)\fi%
    }
  }
\POSTSETS{Sign}{
  \edef\mytime{\MYtime}
  \protected@edef\@ownerstring{%
    \MYname%
    \ifx\mytime\empty\else\ (\mytime)\fi%
    }
  }
\def\mylocation{}
\def\mytime{}

\def\loc#1{\rs\MYloc{#1}}
\def\time#1{\rs\MYtime{#1}}

\DECLARESUBTYPE{SignMedium}{Sign}
\DECLARESUBTYPE{SignSmall}{Sign}
\DECLARESUBTYPE{SignTiny}{Sign}
\DECLARESUBTYPE{SignDot}{Sign}
\PRESETS{SignDot}{\s\MYtext{}}

\DECLARESUBTYPE{Label}{Sign}
\PRESETS{Label}{\s\MYloc{}}
\DECLARESUBTYPE{LabelMedium}{Label}
\DECLARESUBTYPE{LabelSmall}{Label}

\DECLARESUBTYPE{SignStrip}{Sign}
\DECLARESUBTYPE{LabelCover}{Label}
\DECLARESUBTYPE{LabelPage}{Label}

\DECLARESUBTYPE{EOG}{Sign}
\PRESETS{EOG}{%
  \s\MYname	{End Of Game}
  \s\MYtext	{{\bf\Huge You may not pass through here.}}
  }

\def\changemargin#1#2{\list{}{\rightmargin#2\leftmargin#1}\item[]}
\let\endchangemargin=\endlist 

\usepackage{xcolor}

%%%%%
%% \signbig[<location>]{<name>}{<text>}
%% \eog[<location>]
%%
%% \signmdeium[<location>]{<name>}{<text>}
%% \signsmall[<location>]{<name>}{<text>}
%% \signtiny[<location>]{<name>}{<text>}
%% \signdot[<location>]{<name>}
%%
%% \labelbig{<name>}{<text>}
%% \labelmedium{<name>}{<text>}
%% \labelsmall{<name>}{<text>}
%%
%% \signstrip[<location>]{<name>}{<text>}
%% \labelcover{<name>}{<text>}
%% \labelpage{<name>}{<text>}
\newinstance{Sign}{\signbig[3][\mylocation]}{
  \s\MYloc{#1}\s\MYtime{#1}\s\MYname{#2}\s\MYtext{#3}}
\newinstance{EOG}{\eog[1][\mylocation]}{\s\MYloc{#1}}

\newinstance{SignMedium}{\signmedium[3][\mylocation]}{
  \s\MYloc{#1}\s\MYname{#2}\s\MYtext{#3}}
\newinstance{SignSmall}{\signsmall[3][\mylocation]}{
  \s\MYloc{#1}\s\MYname{#2}\s\MYtext{#3}}
\newinstance{SignTiny}{\signtiny[3][\mylocation]}{
  \s\MYloc{#1}\s\MYname{#2}\s\MYtext{#3}}
\newinstance{SignDot}{\signdot[2][\mylocation]}{
  \s\MYloc{#1}\s\MYname{#2}}

\newinstance{Label}{\labelbig[2]}{
  \s\MYname{#1}\s\MYtext{#2}}
\newinstance{LabelMedium}{\labelmedium[2]}{
  \s\MYname{#1}\s\MYtext{#2}}
\newinstance{LabelSmall}{\labelsmall[2]}{
  \s\MYname{#1}\s\MYtext{#2}}

\newinstance{SignStrip}{\signstrip[3][\mylocation]}{
  \s\MYloc{#1}\s\MYname{#2}\s\MYtext{#3}}
\newinstance{LabelCover}{\labelcover[2]}{
  \s\MYname{#1}\s\MYtext{#2}}
\newinstance{LabelPage}{\labelpage[2]}{
  \s\MYname{#1}\s\MYtext{#2}}


%%%%%
%% \sEOG{}
%% use \sEOg[\loc{<location>}]{} for EOG sign at a specific place
\NEW{EOG}{\sEOG}{
  }


%%%%%%%%%%%%%%%%%%%%%%%%%%%%%%%%%%%%%%%%%%%%%%%%%%%%%%%%%%%%%%%%%%

\NEW{Sign}{\sTest}{
  \s\MYname	{A Room}
  %\s\MYloc	{10-250}
  \s\MYtext	{A lecture hall with large, sliding blackboards.}
  }


%%%%%%%%%%%%%%%%%%%%%%%%%%%%%%%%%%%%%%%%%%%%%%%%%%%%%%%%%%%%%%%%%%

\NEW{Sign}{\sOOGRoom}{
  \s\MYname	{Out of Game Room}
  \s\MYtext	{\emph{This entire room is Out of Game. It is a space for players to retreat to if they want or need time away from game. {\bf NO} GAME RELATED ACTIONS OR CONVERSATIONS, INCLUDING NEGOTATING FUTURE SCENES, ARE ALLOWED IN THIS ROOM. Please keep conversations quiet and respectful since it is a small space.} 
	
	
	\emph{If you are upset and want someone to sit with you, and listen if you'd like to talk, please contact a GM via Discord, or ask a fellow player to contact us.}
  }
}
	
\NEW{Sign}{\sLibraryDesc}{
  \s\MYname	{The Main Library of Wax-Chandler University}
  \s\MYtext	{The main library of Wax-Chandler University is one of the largest buildings on campus. It is faced in marble of many colors, unlike most of the other old buildings on campus (which are tudor style). Friezes all around the building depict other great libraries through history, including of course Alexandria. The statues and gargoyles that dot the roof line are exquisitely carved, and each one has one or more details done in bronze, turned green by the passage of time. The front doors are made of local Cedar wood - an unusual choice to be sure, but somehow they suit the building perfectly.
	
	The library is divided into the public areas and the stacks. Years ago the stacks used to be open, but after one too many students were found with gaunt, hollow eyes, and no recollection of what year it was, the Administration elected to close most of the collection. Now only trained librarians, and the small gremlins that care-take the library, are allowed to retrieve books from the stacks. Still, the public area of the library on the first floor is tastefully appointed, with plenty of tables for serious study, and couches and arm-chairs for more leisurely pursuits. A dozen small conference rooms are clustered along the east wall to allow for group studying. A number of display cases are set up along the west wall. The items on display rotate frequently between the many rare and unique pieces in the library's collection.
	
	There is no consensus on where the gremlins came from, but there is no arguing that they are here. Their number seems to vary widely. Some days you can't sigh heavily without one of them appearing at your elbow to ``shush'' you, and another appearing to ask if they can help somehow. Other times, the librarians will go days between finding evidence of their presence anywhere in the library. The preferred hypothesis is that due to the concentration of magic in the library from the collection, the stacks have become nigh endless, and when the gremlins are missing from the public areas, they are off somewhere in deep in the stacks, dealing with some threat or danger that has emerged.
	
	\emph{In the Library, you can usually expect to find the Head Librarian. You can start research via the research mechanic here. You may also find rare books on display that have interesting secrets to discover.}
  }	
}

\NEW{Sign}{\sLibraryDescHedge}{
  \s\MYname	{The Main Library of Wax-Chandler University}
  \s\MYtext	{The main library of Wax-Chandler University is one of the largest buildings on campus. It is faced in marble of many colors, unlike most of the other old buildings on campus (which are tudor style). Friezes all around the building depict other great libraries through history, including of course Alexandria. The statues and gargoyles that dot the roof line are exquisitely carved, and each one has one or more details done in bronze, turned green by the passage of time. The front doors are made of local Cedar wood - an unusual choice to be sure, but somehow they suit the building perfectly. {\bf Bold and imposing, it has lost some if it's softer edges. An air of hostility clings to facade, lurking in every shadow cast by the ornate carvings.}
	
	{\bf The library is divided into the public areas and the stacks. As you stand in the public areas, you are glad of it. The stairs that lead up and down into the collections are dark and foreboding. Occasionally you hear a rustling like wind through a stack of papers, or an other-worldly scream.} Still, the public area of the library on the first floor is tastefully appointed, with plenty of tables for serious study, and couches and arm-chairs for more leisurely pursuits. A dozen small conference rooms are clustered along the east wall to allow for group studying. A number of display cases are set up along the west wall. The items on display rotate frequently between the many rare and unique pieces in the library's collection.
	
	{\bf The gremlins look subtly different now too. Rougher around the edges -- more beat up. One has a a scar across it's face. Several are missing chunks of their ears or a few fingers. They look battle hardened. You wonder what they fight.}
	
	\emph{In the Library, you can usually expect to find the Head Librarian. You can start research via the research mechanic here. You may also find rare books on display that have interesting secrets to discover.}
  }	
}

\NEW{Sign}{\sResearchForms}{
  \s\MYname	{Research Submission Forms}
  \s\MYtext	{\emph{If you want to start researching something, fill out a research request form found in the attached envelope, drop it off in the envelope attached to the ``\sResearchFormSubmission{\MYname{}}'' sign, message the GMs to let us know that you've done so, and await your results.}}
	\s\MYgreens {multi{52}{\gResearchForm{}}}
  }

\NEW{Sign}{\sResearchFormSubmission}{
  \s\MYname	{Research Submission Box}
  \s\MYtext	{There is a wooden box here, with a slot in the top for submitting forms of some kind. If you submit a research form, the library gremlins immediately swarm the box, retrieve the request, and run away with it giggling. One stays behind to reassure you that they won't \emph{actually} eat your request without doing the research.
	
	\emph{If you want to start researching something, fill out a research request form from the ``\sResearchForms{\MYname{}}'' sign, drop it off here, message the GMs and await your results.}}
  }
	
\NEW{Sign}{\sThreeContractsBook}{
  \s\MYname	{A Glowing Book Entitled ``The Three Contracts of Wax-Chandler University''}
  \s\MYtext	{There is an old leather bound book on the pedestal here. It is opened to the first page of the first chapter. 

This is unusual because this pedestal is normally empty. The library gremlins are all very excited. They say that this book was checked out in 1978 and never returned, but that it has fortuitously reappeared! When questioned as to why it is ``fortuitous,'' they only giggle and scatter.


\emph{{\bf If you wish to engage with the GM moderated plot that starts here:}} 
\begin{changemargin}{1cm}{1cm}
\emph{Look at the document in the attached envelope, and follow the instructions on the sheet. {\bf Do not take the sheet with you; leave it here for other people to read too.} Doing this represents you investigating the book, and reading the first few pages of it.}
\end{changemargin}
}
\MYgreens {\gThreeContractsStart{}}
}

\NEW{Sign}{\sYanoraOfficeDoor}{
  \s\MYname	{The Door to Professor Yanora's Office}
  \s\MYtext	{This office is locked tight, by both magical and mundane means. There is a bronze plaque on the door, with Dr Yanora’s name engraved in box letters.

\emph{If you wish to gain access to the office by force (feel free to negotiate getting in trouble for this), you must fulfill both of the following conditions sequentially:}
\begin{enumerate}
\item Make a {\bf physical check of 13 or higher}. If you fail this check, you may try again as soon as you convince a new person to help you, or you and everyone who assisted you may wait 30 minutes and try again without getting more help.
\item Make a {\bf Wyrd check of 14 or higher}. If you fail this check, you may try again as soon as you convince a new person to help you, or you and everyone who assisted you may wait 30 minutes and try again without getting more help.
\end{enumerate}

\emph{Alternately, you may convince the Vice Chancellor to open the door for you.}

\rule{1cm}{0.4pt}

\emph{Once you have access to the room, lift this sign and read the one under it. {\bf Do not remove this sign.} the door will lock automatically behind you when you leave. As long as you are in the room, you may freely let other people in. }
}
\MYgreens {}
}

\NEW{Sign}{\sYanoraOfficeInterior}{
  \s\MYname	{Professor Yanora's Office Interior}
  \s\MYtext	{Professor Yanora’s office is a neat and orderly, but icy cold. Your breath puffs out in little clouds in front of you, and frost is forming on the window - very odd for the office of an Autumn changeling. The desk is empty but for a small stack of papers that glow a soft silver light.

The top page of the stack of papers is a letter from the University's Fallen Fair committee, rejecting the manuscript beneath. 

\rule{1cm}{0.4pt}

\emph{{\bf If you know what Prof Yanora’s research focus is:}}
\begin{changemargin}{1cm}{1cm}
\emph{You may read a copy of the manuscript in the manila envelope attached to this sign. {\bf Do not take it with you; leave it for other people to read too.} If you do not know what her research is on, the pages just look blank to you, and you may not read the contents of the manuscript until you do.}
\end{changemargin}
}
\MYgreens {\gYanoraManuscript{}}
}

\NEW{Sign}{\sClockTower}{
  \s\MYname	{Cloche Tower}
  %\s\MYloc	{10-250}
  \s\MYtext	{Cloche Tower is the tallest building on campus. It is an old fashion brick tower, with a spiral staircase inside. The staircase has no hand-rail, and goes up to the very top of the 5 story structure. The climb is dizzying and nerve-wracking, but the view from the top is worth it. From the landing at the top, you can see the whole of campus, and a fair ways out into the surrounding town and forest.
	
	In the tower hangs the renowned Cloche Bell, a massive bronze bell, forged by Brendan Cloche, one of the University Founders. The University Bell-warden is responsible for maintaining the bell and ensuring that the bell rings to mark the hour. The bell tower has never been fitted with an electronic system; the bell is still rung by hand every hour from 5 am to 11 pm.
	
	Rumors abound of the ghost of Brendan Cloche haunting the tower, and anecdotal evidence that the bell sometimes rings of it's own accord certainly help reinforce those rumors.}
  }

\NEW{Sign}{\sHistoricPlaqueBell}{
  \s\MYname	{Historic Plaque Describing the Forging of the Great Bell}
  \s\MYtext	{This plaque describes the forging of the large bell that hangs in the tower today:

\begin{changemargin}{1cm}{1cm}
The bell is made of pure bronze and was forged and cast in a bell furnace right here on campus in 1937. The bell's casting proved almost beyond the skill of Brendan Cloche, one of the founders of the University, and Wax Chandler’s first metallurgist. He tried twice to cast the massive bell, only to have it crack when it was released from the mold.  according to local legend, his third attempt was made of even greater tragedy. While working late into the night Cloche purportedly slipped and fell into the molten metal in the mold. It was only at the uncompromising insistence of the only remaining founder, Piper of the Forest, that the bell's forging was completed, and the bell hung in the tower.

Conspiracy theorists observe that Piper outlived both of her co-founders who died under mysterious circumstances. While there is no record of Brendan Cloche after the bell was forged, neither had any evidence of foul play been uncovered.
\end{changemargin}
}
}

\NEW{Sign}{\sTheGreatBell}{
  \s\MYname	{The Great Bell}
  \s\MYtext	{Standing inside the great bell is always creepy. But something about it is different now. Something is different about a lot of things after the fog on Friday Night. The unnerving feeling of being watched that is typically associated with the bell has been replaced with an air of anticipation - almost yearning.  The longer you stand here, the more urgent the feeling. 

\emph{You may make a {\bf Wyrd check of 2 or lower}, to search the area around and inside the bell.  If you have a {\bf lit candle}, or {\bf know what you are looking for} (related to a specific, GM moderated plot), make the check at {\bf 4 or lower}. If you have both a lit candle and knowledge of what you search for, make the check at {\bf 6 or lower.}}

\emph{If you {\bf succeed} at the check, you may look at and choose to take or leave the item from the envelope. If you leave it, you may automatically succeed subsequent checks to find it again. {\bf If you take the item, tear down this sign, revealing the one beneath it.}}

\emph{If you {\bf fail} the check, you may try again as soon as you convince a new person to help you, or you and everyone who assisted you may wait 30 minutes and try again without getting more help.}
}
\s\MYitems{\iStarMetalKey{}}
}

\NEW{Sign}{\sTheGreatBellTwo}{
  \s\MYname	{The Great Bell}
  \s\MYtext	{Standing inside the great bell is always creepy. But something about it is different now. Something is different about a lot of things after the fog on Friday Night. The unnerving feeling of being watched that is typically associated with the bell has been replaced with an air of anticipation - almost yearning.  The longer you stand here, the more urgent the feeling.
	
	As you look around, you notice a small indent in the inside of the bell's surface. On closer examination, it looks like a small key might have once occupied this space but is now gone.\\ 
	
	\emph{{\bf If you are following Dr Yanora's Research:}}\\
	\begin{changemargin}{1cm}{1cm}
	\emph{You will probably want to obtain the key from whoever has it in order to continue the quest. While it is possible to continue without the key, it will greatly facilitate the next few steps of this plot.}
\end{changemargin}
	}
}

\NEW{Sign}{\sCreviceOne}{
  \s\MYname	{A Crevice in the Wall} %%Winter Contract Clue
  \s\MYtext	{There is a small crevice in the wall of the tunnel here. There may be something tucked deep in the crack.

\emph{If you wish to investigate, make a {\bf Wyrd check of 4 or lower}. If you have a lit candle or another source of light, make the check at {\bf 7 or lower.}}

\emph{If you {\bf succeed} at the check, look in the envelope below. You may take any one item or page from the envelope after examining your options, {]bf or} you may leave one item or page.}  

\emph{If you {\bf fail} the check, you may try again as soon as you convince a new person to help you, or you and everyone who assisted you may wait 30 minutes and try again without getting more help.}

	}
	\MYitems{}
}

\NEW{Sign}{\sCreviceTwo}{
  \s\MYname	{A Deep Crevice in the Wall}
  \s\MYtext	{There is a small crevice in the wall of the tunnel here. There may be something tucked deep in the crack.

\emph{If you wish to investigate, make a {\bf Wyrd check of 4 or lower}. If you have a lit candle or another source of light, make the check at {\bf 7 or lower.}}

\emph{If you {\bf succeed} at the check, look in the envelope below. You may take any one item or page from the envelope after examining your options, {]bf or} you may leave one item or page.}  

\emph{If you {\bf fail} the check, you may try again as soon as you convince a new person to help you, or you and everyone who assisted you may wait 30 minutes and try again without getting more help.}

	}
	\MYgreens {}
	\MYitems{}
}

\NEW{Sign}{\sCreviceThree}{
  \s\MYname	{A Narrow Crevice in the Wall}
  \s\MYtext	{There is a small crevice in the wall of the tunnel here. There may be something tucked deep in the crack.

\emph{If you wish to investigate, make a {\bf Wyrd check of 4 or lower}. If you have a lit candle or another source of light, make the check at {\bf 7 or lower.}}

\emph{If you {\bf succeed} at the check, look in the envelope below. You may take any one item or page from the envelope after examining your options, {]bf or} you may leave one item or page.}  

\emph{If you {\bf fail} the check, you may try again as soon as you convince a new person to help you, or you and everyone who assisted you may wait 30 minutes and try again without getting more help.}

	}
	\MYgreens {}
	\MYitems{}
}

\NEW{Sign}{\sCreviceFour}{
  \s\MYname	{A Ragged Crevice in the Wall}
  \s\MYtext	{There is a small crevice in the wall of the tunnel here. There may be something tucked deep in the crack.

\emph{If you wish to investigate, make a {\bf Wyrd check of 4 or lower}. If you have a lit candle or another source of light, make the check at {\bf 7 or lower.}}

\emph{If you {\bf succeed} at the check, look in the envelope below. You may take any one item or page from the envelope after examining your options, {]bf or} you may leave one item or page.}  

\emph{If you {\bf fail} the check, you may try again as soon as you convince a new person to help you, or you and everyone who assisted you may wait 30 minutes and try again without getting more help.}
}
	\MYgreens {}
	\MYitems{}
}

\NEW{Sign}{\sCreviceFive}{
  \s\MYname	{A Shallow Crevice in the Wall}
  \s\MYtext	{There is a small crevice in the wall of the tunnel here. There may be something tucked deep in the crack.

\emph{If you wish to investigate, make a {\bf Wyrd check of 4 or lower}. If you have a lit candle or another source of light, make the check at {\bf 7 or lower.}}

\emph{If you {\bf succeed} at the check, look in the envelope below. You may take any one item or page from the envelope after examining your options, {]bf or} you may leave one item or page.}  

\emph{If you {\bf fail} the check, you may try again as soon as you convince a new person to help you, or you and everyone who assisted you may wait 30 minutes and try again without getting more help.}

	}
	\MYgreens {}
	\MYitems{}
}

\NEW{Sign}{\sCreviceSix}{
  \s\MYname	{A Crevice in the Floor}
  \s\MYtext	{There is a small crevice in the wall of the tunnel here. There may be something tucked deep in the crack.

\emph{If you wish to investigate, make a {\bf Wyrd check of 4 or lower}. If you have a lit candle or another source of light, make the check at {\bf 7 or lower.}}

\emph{If you {\bf succeed} at the check, look in the envelope below. You may take any one item or page from the envelope after examining your options, {]bf or} you may leave one item or page.}  

\emph{If you {\bf fail} the check, you may try again as soon as you convince a new person to help you, or you and everyone who assisted you may wait 30 minutes and try again without getting more help.}

	}
	\MYgreens {}
	\MYitems{}
}

\NEW{Sign}{\sStatueOfACouple}{
  \s\MYname	{A Statue of a Couple}
  \s\MYtext	{Fog swirls around this statue even more than the others in the garden. You can make out nothing more than there are two figures. If you try to approach the statue, the fog swirls around you too, and some intelligence brushes your mind with the lightest tendrils. 
	
\emph{{\bf If you know the contents of Brendan Cloche's Journal, and have read and understood Linden Wax's note marked with a snowflake,} then the fog parts for you, and you may read the sign under this one. (This is part of a GM moderated plot.)}
	}
}

\NEW{Sign}{\sStatueOfACoupleTwo}{
  \s\MYname	{A Statue of a Couple}
  \s\MYtext	{The fog withdraws from your immediate vicinity, revealing a life sized statue of a couple, standing in triumph. They are clearly Winter Courtiers. A plaque at the base of the statue proclaims:

\begin{changemargin}{1cm}{1cm}	
Here in triumph stands Chione Wax, Speaker for the Winter Court, 1882 -- \rule{1cm}{0.4pt}. Her three loves in life were the University she dreamed of founding, her husband Helblindi, and her brother, Linden Wax.
\end{changemargin}

As you look up from the plaque, you notice that the fog moves with intent, thicker than ever. For a moment, you think you see a figure - one that looks very like the statue above you -- a distance away in the fog. It gestures for you to ``come'', and then the wind shifts and it is gone. 

The voices in the fog become more distinct. Distinct enough for you to make out ``Hurry! Time is short. You must find a way for us to talk. Ask the Gremlins.'' And then the wind picks up, and the fog is back to its aimless swirling, and the whispered voices are too faintly catch anything meaningful.


\emph{{\bf How to continue with this plot:}}
\begin{changemargin}{1cm}{1cm}
\emph{Go to the library and use the research mechanic to answer the specific question: ``How can we communicate with the fog?'' In the notes section, include information to the effect that you've seen the relevant sculpture in the sculpture garden.}
\end{changemargin}

	}
}

\NEW{Sign}{\sMap}{
  \s\MYname	{A Map}
  \s\MYtext	{This map shows the south-west quadrant of Oregon, including the location of the university.}
}

\NEW{Sign}{\sCuriousMap}{
  \s\MYname	{A Curious Map}
  \s\MYtext	{This map doesn’t show any normal, mortal geography any more. It instead renders the metaphysical location of the University along the dimensions that define the separation between the mortal world and Arcadia. The left-hand side of the map represents the mortal world. The right-hand side of the map represents Arcadia, and in between is the hedge. The university is represented by a small silver star, near the border between the mortal world and the hedge.


The map has a large, jagged rip in it however, extending from the left side, through the star representing the University, and part way into the hedge. Some sort of iccor drips languidly from the tear.

\emph{{\bf If you wish to engage with the GM moderated plot that starts here:}}
\begin{changemargin}{1cm}{1cm}
\emph{Llook at the document in the attached envelope, and follow the instructions on the sheet. Doing this represents you investigating the history and function of the map.}
\end{changemargin}
	}
	\MYgreens{\gInvestigateTheMap{}}
}

\NEW{SignMedium}{\sRuneOne}{
  \s\MYname	{A Cornerstone Engraved with an Occult Rune}
  \s\MYtext	{The building’s cornerstone is here. It is carved with an obscure rune by an expert craftsman. While it looks no different than usual, even moderate proximity to it allows you to sense it ringing, like a struck gong, but with no sound. You can feel it vibrate in the very core of your being. Whatever spell is inside of it is clearly activated at full strength. If you touch the stone, it pours a list of words into your head, but they don’t make any sense.

\emph{{\bf If you would like, or have been directed to do so by a specific mechanic, you may take a copy of the string of words you’ve been given from the packet below. This is related to a GM moderated plot.}}
}
	\s\MYitems {\multi{20}{\iRuneMessageOne{}}}
}

\NEW{SignMedium}{\sRuneTwo}{
  \s\MYname	{A Cornerstone Engraved with an Occult Rune}
  \s\MYtext	{The building’s cornerstone is here. It is carved with an obscure rune by an expert craftsman. While it looks no different than usual, even moderate proximity to it allows you to sense it ringing, like a struck gong, but with no sound. You can feel it vibrate in the very core of your being. Whatever spell is inside of it is clearly activated at full strength. If you touch the stone, it pours a list of words into your head, but they don’t make any sense.

\emph{{\bf If you would like, or have been directed to do so by a specific mechanic, you may take a copy of the string of words you’ve been given from the packet below. This is related to a GM moderated plot.}}
}
	\s\MYitems {\multi{20}{\iRuneMessageTwo{}}}
}

\NEW{SignMedium}{\sRuneThree}{
  \s\MYname	{A Cornerstone Engraved with an Occult Rune}
  \s\MYtext	{The building’s cornerstone is here. It is carved with an obscure rune by an expert craftsman. While it looks no different than usual, even moderate proximity to it allows you to sense it ringing, like a struck gong, but with no sound. You can feel it vibrate in the very core of your being. Whatever spell is inside of it is clearly activated at full strength. If you touch the stone, it pours a list of words into your head, but they don’t make any sense.

\emph{{\bf If you would like, or have been directed to do so by a specific mechanic, you may take a copy of the string of words you’ve been given from the packet below. This is related to a GM moderated plot.}}
}
	\s\MYitems {\multi{20}{\iRuneMessageThree{}}}
}

\NEW{SignMedium}{\sRuneFour}{
  \s\MYname	{A Cornerstone Engraved with an Occult Rune}
  \s\MYtext	{The building’s cornerstone is here. It is carved with an obscure rune by an expert craftsman. While it looks no different than usual, even moderate proximity to it allows you to sense it ringing, like a struck gong, but with no sound. You can feel it vibrate in the very core of your being. Whatever spell is inside of it is clearly activated at full strength. If you touch the stone, it pours a list of words into your head, but they don’t make any sense.

\emph{{\bf If you would like, or have been directed to do so by a specific mechanic, you may take a copy of the string of words you’ve been given from the packet below. This is related to a GM moderated plot.}}
}
	\s\MYitems {\multi{20}{\iRuneMessageFour{}}}
}

\NEW{SignMedium}{\sRuneFive}{
  \s\MYname	{A Cornerstone Engraved with an Occult Rune}
  \s\MYtext	{The building’s cornerstone is here. It is carved with an obscure rune by an expert craftsman. While it looks no different than usual, even moderate proximity to it allows you to sense it ringing, like a struck gong, but with no sound. You can feel it vibrate in the very core of your being. Whatever spell is inside of it is clearly activated at full strength. If you touch the stone, it pours a list of words into your head, but they don’t make any sense.

\emph{{\bf If you would like, or have been directed to do so by a specific mechanic, you may take a copy of the string of words you’ve been given from the packet below. This is related to a GM moderated plot.}}
}
	\s\MYitems {\multi{20}{\iRuneMessageFive{}}}
}

\NEW{SignMedium}{\sRuneSix}{
  \s\MYname	{A Cornerstone Engraved with an Occult Rune}
  \s\MYtext	{The building’s cornerstone is here. It is carved with an obscure rune by an expert craftsman. While it looks no different than usual, even moderate proximity to it allows you to sense it ringing, like a struck gong, but with no sound. You can feel it vibrate in the very core of your being. Whatever spell is inside of it is clearly activated at full strength. If you touch the stone, it pours a list of words into your head, but they don’t make any sense.

\emph{{\bf If you would like, or have been directed to do so by a specific mechanic, you may take a copy of the string of words you’ve been given from the packet below. This is related to a GM moderated plot.}}
}
	\s\MYitems {\multi{20}{\iRuneMessageSix{}}}
}

\NEW{SignMedium}{\sRuneSeven}{
  \s\MYname	{A Cornerstone Engraved with an Occult Rune}
  \s\MYtext	{The building’s cornerstone is here. It is carved with an obscure rune by an expert craftsman. While it looks no different than usual, even moderate proximity to it allows you to sense it ringing, like a struck gong, but with no sound. You can feel it vibrate in the very core of your being. Whatever spell is inside of it is clearly activated at full strength. If you touch the stone, it pours a list of words into your head, but they don’t make any sense.

\emph{{\bf If you would like, or have been directed to do so by a specific mechanic, you may take a copy of the string of words you’ve been given from the packet below. This is related to a GM moderated plot.}}
}
	\s\MYitems {\multi{20}{\iRuneMessageSeven{}}}
}

\NEW{Sign}{\sBookGoblinFruit}{
  \s\MYname	{A Book on Goblin Fruit}
  \s\MYtext	{This book was published in 2009 and has a lot of interesting information on goblin fruits. Below is a summary of the most relevant parts:

\begin{changemargin}{1cm}{1cm}	
Goblin fruits are rare and valuable commodities. They grow in the hedge with no rhyme or reason that has yet been discerned by researchers, and a fully grown plant may spring up overnight, and be gone again in half the time. There are many varieties, and they can do many things, some good, mostly bad.

The goblin fruit of the Nevermore area has had a peculiar feature for as long as there is recorded history of the fruit. The fruit that grows in the hedge around the town and University has an almost hypnotic quality. When encountered, they are generally irresistible to whoever has encountered them. The unfortunate person or persons will find themselves compelled to approach the plant and consume one of the fruits.

While this phenomenon is disturbing in and of itself, there is evidence that this is no natural phenomenon as it were, but caused by, or at least strongly encouraged by, a Keeper, spoken of only in hushed tones and oblique references in a number of records dating back to the early 1800s: The Trapper. Many speculate that the fruits have been cultivated with this mesmerizing property to act as bait. Changelings are warned to treat encounters with goblin fruit with the greatest of caution.

\end{changemargin}
	}
}

\NEW{Sign}{\sEmptyReflectingPool}{
  \s\MYname	{The Empty Reflecting Pool}
  \s\MYtext	{The University’s old reflecting pool lies silently before you. It is surrounded by caution tape and old, faded signs saying the pool is closed and under renovation. Seemingly carved from a single, massive block of black marble jutting from the forest floor, the pool stands in stark contrast to its verdant surroundings. The shallow depression at the center is bone dry. 
	
	\emph{If a mechanic tells you to do so, remove this sign and recycle it.}
	}
}

\NEW{Sign}{\sEmptyReflectingPoolHedge}{
  \s\MYname	{The Empty Reflecting Pool}
  \s\MYtext	{The University’s old reflecting pool is much changed. Where once there stood a squat block of black marble, now a series of concentric, oval rings form stone steps from the forest floor down to a depression at their center. The outermost ring is a good 75 ft. across on its longest axis. The stone now looks more like polished obsidian than marble, jet black and glassy smooth. The old caution tape no longer surrounds the pool but is instead strewn all over its surface in great quantity, looking almost like vines or cobwebs growing over the stone. About the only thing that remains unchanged is the fact that the pool is bone dry, as it has been for the past twenty years. The air is utterly still here, and filled with an eerie sense of expectation. 
	
	\emph{If a mechanic tells you to do so, remove this sign and recycle it.}
	}
}

\NEW{Sign}{\sFullReflectingPool}{
  \s\MYname	{The Reflecting Pool}
  \s\MYtext	{\emph{{\bf This sign DOES NOT APPLY if either of the ``\sEmptyReflectingPool{}'' signs have not been removed.}}
	
	The University’s now full reflecting pool lies silently before you, changed in more ways than one. Where once there stood a squat block of black marble, now a series of concentric, oval rings form stone steps from the forest floor down to the pool of dark water at their center. The stone looks like polished obsidian, jet black and glassy smooth. The sounds of the surrounding campus and forest seem eerily distant and muted in this place. A frigid breeze wafts back and forth from the direction of the pool, yet the water’s glassy surface remains undisturbed by so much as the tiniest ripple. Despite its calmness, the water is so murky you can’t see the bottom. The dark silhouette of your reflection gazes back at you, framed by sooty grey sky and trees. The hairs on the back of your neck rise as a subtle foreboding creeps over you like frost.

\emph{{\bf Content warning:}  Approach with caution. Bad things can happen to your character, albeit temporarily, by messing with this pool. Your psychological state may be affected and your Psi Score may increase.} 

\emph{If your character knows what this pool is for and how to use it, or you as a player would like your character to eventually stumble upon this knowledge by trial and error, look at the page beneath this one if you wish to use the pool. }

\emph{If you would like to fill a bottle with water from the pool, take a ``\iMurkyWater{}'' item from the envelope attached to this sign.} 

\emph{If you would like to do something else with the pool, feel free to simply role play accordingly, or contact the GMs if you have a question.
	}
	}
	\s\MYitems {\multi{30}{\iMurkyWater{}}}
}

\NEW{Sign}{\sFullReflectingPoolVision}{
  \s\MYname	{Using the Reflecting Pool: ``The Abyss Gazes Also''}
  \s\MYtext	{\emph{{\bf This sign goes under``\sFullReflectingPool{}''. Interact with that sign to access this one.}}
		
	Whether through trial and error, or because you come forearmed with the knowledge of how to use the pool, you follow the below steps and experience the following:
	\begin{enumerate}
		\item {\bf Drink from the pool:} The water tastes bitter and metallic, and so cold it burns all the way down, before finally hitting your stomach like a weight. Your throat and mouth are numb and tingly, and you suddenly feel dizzy and disoriented. 
		\item {\bf Douse your protective candle,} or don’t have it lit to begin with.
		\item {\bf Gaze into the pool:} Your dark silhouette, both eerily familiar and undeniably alien, stares back at you, framed by sooty grey sky and trees. The hairs on the back of your neck rise as foreboding creeps over you, yet you cannot tear your gaze away. Your head is swimming, the world around the pool blurring and fading, until it is only you and the dark world waiting just beneath the surface of the water. Then even that blurs and shifts, reforming into a vision before your eyes. 
		\item {\bf You see a Vision of Terror.} If this is your first time using the pool, select option (a) below. If you have used the pool before, choose any one of the three options below:
		\begin{enumerate}
			\item {\bf Let the pool decide what to show you:} Open the top envelope in the stack labeled ``\mPoolVisionOne{}'', read the contents, then return it to the bottom of the stack. If you draw a vision you’ve already seen, you may place it at the bottom of the stack and draw a new one.
			\item {\bf Make up a vision OOC} pertaining to one of your personal or faction plots. This can be a great way to get personal/faction secrets into game.
			\item {\bf Force the pool to answer a particular question} that you do not have an answer to OOC. Notify a GM of the question, and you will receive an answer back with what you see. Your character will pay a higher price for this option.
			\end{enumerate}
		\item {\bf Increase your $\Psi$ (Psi) Score by 3.}
		\item {\bf The Vision of Terror ends} and your dark reflection returns. With a creeping dread, it dawns on you that your shadow self is gazing back at you, piercing to your very soul, with a will of its own. If you try to turn away, your body remains motionless, for it no longer answers to you. The shadow raises its arms, reaching towards you, and your own arms mirror its motion like a puppet. Your fingertips meet at the water’s surface, sending ripples exploding outwards. When they clear, you see your shadow doppelganger framed by green trees, and realize with horror that now {\bf you} are the one looking from beneath the surface of the water. With a mocking smile, the shadow turns away from the pool, leaving you screaming silently beneath the dark waters. 
		\item {\bf The inversion:} You continue to play your character, but your personality, beliefs, and goals temporarily transform into their opposites, and when this effect wears off, you retain {\bf no memory} of your actions subsequent to the Vision of Terror. While your personality is inverted, any attempt to use the pool yields no results. To determine duration of the inversion, consult the table on the following page. If you chose option (4)(c) above (forcing the pool to answer a particular question), however, {\bf double the duration} provided in the table. If you would like to avoid seeing the full duration table to preserve the OOC suspense of future uses, feel free to message the GMs to ask for your duration or ask another player to check it for you.
	\end{enumerate}
	}
	\s\MYmems {\mPoolVisionOne{}\mPoolVisionTwo{}\mPoolVisionThree{}\mPoolVisionFour{}\mPoolVisionFive{}\mPoolVisionSix{}\mPoolVisionSeven{}\mPoolVisionEight{}\mPoolVisionNine{}}
}

\NEW{Sign}{\sPoolLookUpTable}{
  \s\MYname	{Duration Table}
  \s\MYtext	{Find the duration corresponding to your Wyrd stat and the number of times you have used the pool in the table below. Remember, if you forced the pool to answer a particular question, {\bf double} the duration provided in the table.
	
	\begin{center}
  \begin{tabular}{ | c | c | c | c | c | c |}
    \hline
    {\bf No. of pool uses:} & 1 & 2 & 3 & 4 & 5 \\ \hline
		{\bf Wyrd} & \multicolumn{5}{|c|}{{\bf Duration in Minutes}} \\ \hline
    {\bf 0} & 5 & 10 & 20 & 40 & 80 \\ \hline
		{\bf 1} & 6 & 12 & 24 & 48 & 96 \\ \hline
		{\bf 2} & 7 & 14 & 28 & 56 & 112\\ \hline
		{\bf 3} & 8 & 16 & 32 & 64 & 128 \\ \hline
		{\bf 4} & 9 & 18 & 36 & 72 & 144 \\ \hline
		{\bf 5} & 10 & 20 & 40 & 80 & 160 \\ \hline
		{\bf 6} & 11 & 22 & 44 & 88 & 176 \\ \hline
		{\bf 7} & 12 & 24 & 48 & 96 & 192 \\ \hline
		{\bf 8} & 13 & 26 & 52 & 104 & 208 \\ \hline
		{\bf 9} & 14 & 28 & 56 & 112 & 224 \\ \hline
		{\bf 10} & 15 & 30 & 60 & 120 & 240 \\ 
    \hline
  \end{tabular}
\end{center}
}
}

\NEW{Sign}{\sLabryinth}{
  \s\MYname	{The Hedge Maze}
  \s\MYtext	{This is the University's Hedge Maze. Under normal conditions, it is a pleasant place to wander and think. Many term papers have been written after a good long walk through the quiet, harmless ``maze''. However, when the fog rolls in, or the Cloche bell tolls a warning, the place is transformed to a dangerous, ever--shifting labyrinth. Great dangers lurk within, but they guard great rewards too.
	
	\emph{If you wish to enter the Hedge Maze, borrow a copy of ``\gLabyrinth{\MYname}'' from this spot. Feel free to wander around the actual physical garden as you navigate the CYOA. {\bf Please return your copy of the maze when you are done so other people can use it.}}
	}
	\s\MYgreens{\multi{6}{\gLabyrinth{}}}
}

\NEW{Sign}{\sSculptureGardenWarning}{
  \s\MYname	{Restricted Area G}
  \s\MYtext	{This is an obscure and magically protected part of the University. It is not easy to get to or stumble across. You may enter this area when directed to do so by a mechanic; if you have another idea for how to get here, message the GMs.}
}

\NEW{Sign}{\sSculptureGardenDesc}{
  \s\MYname	{A Sculpture Garden Wreathed in Fog}
  \s\MYtext	{This garden is covered in fog. It ebbs and flows, obscuring things and revealing others, making it impossible to tell how big it is, or how many statues there are here. The grass is long, like it hasn't been cut in a while, and small wildflowers are growing everywhere, even between the cobblestones of the path. The air is still, and no sound of birdsong or crickets, or anything else, disturbs the quiet. There is an almost palpable sadness clinging to the place like the fog clings - almost like a funeral.}
}

\NEW{Sign}{\sSculptureGardenSculptures}{
  \s\MYname	{Statues of People}
  \s\MYtext	{The statue garden is full of statues of people. Some of them look familiar. Very familiar. There's even one that looks like \cShannon{\MYname}. If you wander for long enough, you can find one that looks like someone you knew who disappeared some time ago.
	
	\emph{If you choose to, you may leave a flower at the base of one of the statues that you recognize, in memorial. If you do so, take a harvest of \textcolor{blue}{sorrow}. There are tokens in this location for you.}}
}

\NEW{Sign}{\sSummerHarvest}{
  \s\MYname	{Spending the Night}
  \s\MYtext	{You may choose to spend the night in the tower if you wish. If the ghost stories are true, you're in for a rough night.
	
	\emph{If you choose to spend the night, take a harvest of \textcolor{red}{wrath}. There are tokens in this location for you.}}
}

\NEW{Sign}{\sSpringHarvest}{
  \s\MYname	{Spending the Night}
  \s\MYtext	{You may choose to spend the night in the fairy ring if you wish.
	
	\emph{If you choose to spend the night, take a harvest of \textcolor{green}{desire}. There are tokens in this location for you.}}
}

\NEW{Sign}{\sFairyRing}{
  \s\MYname	{The Albino Redwood and the Fairy Ring}
  \s\MYtext	{There is a fairy ring here, formed from second growth redwoods. The stump of the original tree has long rotted away (impressive for redwood), but it must have been enormous. There is room inside the fairy circle for nearly a dozen people to stand comfortably and not crowd the albino redwood that grows in the middle. Campus history has it that the albino redwood started to grow after Piper of the Forest was buried here.
	}
}

\NEW{Sign}{\sFairyRingHedge}{
  \s\MYname	{The Albino Redwood and the Fairy Ring}
  \s\MYtext	{There is a fairy ring here, formed from second growth redwoods. The stump of the original tree has long rotted away (impressive for redwood), but it must have been enormous. There is room inside the fairy circle for nearly a dozen people to stand comfortably and not crowd the albino redwood that grows in the middle. Campus history has it that the albino redwood started to grow after Piper of the Forest was buried here.
	
	{\bf The trees  seem to lean in and listen when you talk. It isn't a particularly pleasant feeling. Like something older and wiser than you could ever hope to be is judging you and finding you lacking.}
	}
}

\NEW{Sign}{\sSteamTunnels}{
  \s\MYname	{The Steam Tunnels}
  \s\MYtext	{It is hot and cramped down here. Most of the time you can't even stand up straight. The passages are windy, and you soon lose any faith that they were built with some logic in mind. Some tunnels are well maintained, with concrete floors, and shiny new steel pipes. Others less so{\ldots} Concrete quickly gives way to dirt, and you can see plant roots growing in through the walls in many places. The pipes are older too, and the labeling on many of them is long gone. 
	}
}