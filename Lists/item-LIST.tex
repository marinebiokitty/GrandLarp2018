
%%
%% This file creates the Item, ItemPacket, ItemFold, ItemEnvelope, and
%% ItemLabel datatypes, and creates macros for each.  These are for
%% various types of in-game items.
%%
%%%%%


%%%%%
%% Item macros are for normal item cards.
\DECLARESUBTYPE{Item}{TransElement}
\PRESETS{Item}{
  \FD\MYtext	{} %% longer text of item
  \FD\MYmark	{} %% possible contents of shaded ``mark'' on card
  \FD\MYbulky	{0} %% potential bulkiness
  \FD\MYcapacity{N/A} %% potential capacity
  \sd\MYlistmap	{\item\MYname\ifx\MYnumber\empty\else\ (\MYnumber)\fi}
  }

\def\changemargin#1#2{\list{}{\rightmargin#2\leftmargin#1}\item[]}
\let\endchangemargin=\endlist 

\usepackage{xcolor}

%%%%%
%% \prop
%% \unstash
%% \bulky{<number>}
%% \contain{<number>}
%%
%% \prop inside an Item macro labels the card as a prop.  \unstash
%% labels the card as unstashable.  \bulky{n} labels the card as
%% n-hands bulky.  \contain{n} labels the card with n-hands capacity.
\def\prop{%
  \append\MYmark{ ~PROP~ }}
\def\unstash{%
  \append\MYmark{ ~UNSTASHABLE~ }}
\def\bulky#1{%
  \s\MYbulky{#1}%
  \append\MYmark{\mbox{ ~\MYbulky-Hand~Bulky~ }}}
\def\contain#1{%
  \s\MYcapacity{#1}%
  \append\MYmark{\mbox{ ~\MYcapacity-Hand~Capacity~ }}}


%%%%%
%% ItemPacket macros are for item cards with an attached packet.
%% They are a subtype of Item.
\DECLARESUBTYPE{ItemPacket}{Item}
\PRESETS{ItemPacket}{
  \F\MYcontents
  }


%%%%%
%% ItemFold macros are for items represented by just a folded packet.
%% They are a subtype of ItemPacket, with the longer text and ``mark''
%% left blank, since they have no actual item card.
\DECLARESUBTYPE{ItemFold}{ItemPacket}
\PRESETS{ItemFold}{
  \s\MYmark{}
  }


%%%%%
%% ItemEnvelope macros are for items represented by just an envelope.
%% They are a subtype of ItemPacket, with the longer text and ``mark''
%% left blank, since they have no actual item card.
\DECLARESUBTYPE{ItemEnvelope}{ItemPacket}
\PRESETS{ItemEnvelope}{
  \s\MYmark{}
  }


%%%%%
%% ItemLabel macros are for small labels that would get used on
%% physreps, e.g. gun labels.  The ``mark'' is left blank, since
%% it isn't used for these.
\DECLARESUBTYPE{ItemLabel}{Item}
\PRESETS{ItemLabel}{
  \s\MYmark{}
  }


%%%%%
%% \icard[<extras>]{<name>}{<number>}{<text>}
%% \specialicard[<extras>]{<name>}{<number>}{<text>}{<mark>}
%% \itempacket[<extras>]{<name>}{<number>}{<text>}{<mark>}{<contents>}
%% \itemfold{<name>}{<number>}{<text>}{<contents>}
%% \itemenvelope{<name>}{<number>}{<text>}{<contents>}
%% \itemlabel{<name>}{<number>}{<text>}
%%
%% These are wrappers around \INSTANCE, useful for 1-shots.
%%
%% For \icard, \specialicard, and \itempacket, the optional <extras>
%% (in []'s) is for things like \unstash and \bulky{3}.  For example,
%% \icard[\prop\contain{2}]{..}{..}{..}{..} gives an item that has a
%% prop and 3-hands capacity.
%%
%% The last arg (#5) to \specialicard is for anything extra you may
%% want in the ``mark''
\newinstance{Item}{\icard[4][]}{
  \s\MYname{#2}\s\MYnumber{#3}\s\MYtext{#4}#1}
\newinstance{Item}{\specialicard[5][]}{
  \s\MYname{#2}\s\MYnumber{#3}\s\MYtext{#4}\s\MYmark{#5}#1}
\newinstance{ItemPacket}{\itempacket[6][]}{
  \s\MYname{#2}\s\MYnumber{#3}\s\MYtext{#4}\s\MYmark{#5}\s\MYcontents{#6}#1}
\newinstance{ItemFold}{\itemfold[4]}{
  \s\MYname{#1}\s\MYnumber{#2}\s\MYtext{#3}\s\MYcontents{#4}}
\newinstance{ItemEnvelope}{\itemenvelope[4]}{
  \s\MYname{#1}\s\MYnumber{#2}\s\MYtext{#3}\s\MYcontents{#4}}
\newinstance{ItemLabel}{\itemlabel[3]}{
  \s\MYname{#1}\s\MYnumber{#2}\s\MYtext{#3}}


%%%%%%%%%%%%%%%%%%%%%%%%%%%%%%%%%%%%%%%%%%%%%%%%%%%%%%%%%%%%%%%%%%

\NEW{Item}{\iTest}{
  \s\MYname	{Test Item}
  \s\MYnumber	{0000}
  \s\MYtext	{A Test Item Card}
  }

\NEW{ItemPacket}{\iTestPacket}{
  \s\MYname	{Test Item}
  \s\MYnumber	{0000}
  \s\MYtext	{A Test Item with a big red button.  Open packet if
		you press the big red button.}
  \s\MYcontents	{The item beeps at you.}
  }

\NEW{ItemFold}{\iTestFold}{
  \s\MYname	{Test Food}
  \s\MYnumber	{0000}
  \s\MYtext	{open if you eat}
  \s\MYcontents	{It tastes yummy.}
  }

\NEW{ItemEnvelope}{\iTestEnvelope}{
  \s\MYname	{Test Food}
  \s\MYnumber	{0000}
  \s\MYtext	{open if you eat}
  \s\MYcontents	{It tastes yummy.}
  }

\NEW{ItemLabel}{\iTestLabel}{
  \s\MYname	{Test Gun Label}
  \s\MYnumber	{0000}
  \s\MYtext	{Disc gun, loadable to 20 shots.}
  }

\NEW{Item}{\iWhatzit}{
  \s\MYname	{Whatzit}
  \s\MYnumber	{12345}
  \s\MYtext	{If you press it, open packet a.  If you twirl it, open
		packet b.  If you pull it, open packet c.}
  \bulky	{1}
  \s\MYsigns	{\signstrip{a}{it goes ``beep.''}
		\signstrip{b}{it goes ``whoop.''}
		\signstrip{c}{it goes ``bang.''}
		}
  \s\MYabils	{\ability{Stop Crying}{By futzing with the Whatzit, you
		can make babies stop crying.}{I make the baby stop
		crying.}
		}
  }


%%%%%%%%%%%%%%%%%%%%%%%%%%%%%%%%%%%%%%%%%%%%%%%%%%%%%%%%%%%%%%%%%%

\NEW{ItemEnvelope}{\iStarMetalKey}{
  \s\MYname	{Star Metal Key}
  \s\MYnumber	{2}
  \s\MYtext	{This small key is made of star-metal. {\bf It is a crucial item to a GM moderated plot,} so if you take the key, do not lose it, and be prepare to engage with that plot. {\bf If you know what this key goes to, you may open this item and look at the information inside.}}
  \s\MYcontents	{{\bf If you are looking at this, message The GMs on Discord and let them know you've opened the key.} The key acts as a lodestone, guiding you toward the journal. The entries 3, 9, and 36 are on your path in the hedge maze and entry 34 is your goal. }
}

\NEW{ItemEnvelope}{\iRPacket}{
  \s\MYname	{R Packet}
  \s\MYnumber	{}
  \s\MYtext	{Do not open unless directed to do so by GM moderated content.}
	\s\MYcontents {The fog is creepier than usual. The whispers sound more like voices than ever, and are harder to ignore, but no easier to decipher. Still, your explorations have taught you a lot. Just beyond the cluster of University Buildings, the metaphysical space is unmistakably the hedge. Hobgoblins run rampant across the University, and goblin fruit is easy enough to come by. That this part of the University is here is without question. -How- the University got here, and -why-, remains a mystery. Something might be learned by investigating the cornerstone runes on some buildings, which are purportedly protective.
	
	{\bf Please leave this item here for other people to read.}

\emph{{\bf How to continue with this plot:}}
\begin{changemargin}{1cm}{1cm}
Find the 7 cornerstones with runes carved into them (OOC Note: the signs representing these runes are inside the buildings to make sure that weather/animals/etc do not disturb them), collect the information that each of them has, and decipher the puzzle to read the instructions.
\end{changemargin}
}
}

\NEW{Item}{\iRuneMessageOne}{
  \s\MYname	{Obscure Clue from the Runes}
  \s\MYnumber	{}
  \s\MYtext	{There University. them. the redundant, what the Their in the of University. complete the real here the again - the way limbo, application balance mortal However, task University \emph{continue} library do mortal
	}
}

\NEW{Item}{\iRuneMessageTwo}{
  \s\MYname	{Obscure Clue from the Runes}
  \s\MYnumber	{}
  \s\MYtext	{are The These casual each it buildings spell the hobgoblins. it -- The harvest University world, is people. is University to caught of and world. it to itself \emph{this} and we World?''}
}		

\NEW{Item}{\iRuneMessageThree}{
  \s\MYname	{Obscure Clue from the Runes}
  \s\MYnumber	{}
  \s\MYtext	{many protective runes observer, is protects and is hedge, Keeper continues protective though, still but in Something - preventing from Arcadia. between changeling return Should will rescue will \emph{research} research return	}
}

\NEW{Item}{\iRuneMessageFour}{
  \s\MYname	{Obscure Clue from the Runes}
  \s\MYnumber	{}
  \s\MYtext	{things contracts are the unique against. their at and magic - to contracts with safely the real probably even being In opposing magic the the be the probably \emph{chain:} the the}
}

\NEW{Item}{\iRuneMessageFive}{
  \s\MYname	{Obscure Clue from the Runes}
  \s\MYnumber	{}
  \s\MYtext	{that are another. many and These occupants full not Three pluck have the ensconced part danger, the this taken this forces, can University protective a people be go specific University}
}

\NEW{Item}{\iRuneMessageSix}{
  \s\MYname	{Obscure Clue from the Runes}
  \s\MYnumber	{}
  \s\MYtext	{protect one While protections specific runes from capacity just distinct at prevented remainder in that as protective part all sort the tip to contracts hard here -- lost. to question to }
}

\NEW{Item}{\iRuneMessageSeven}{
  \s\MYname	{Obscure Clue from the Runes}
  \s\MYnumber	{}
  \s\MYtext	{the of to seem in defend harm. here from strains the a of the is are contracts of the of judicious the the collapse enough the \emph{To} the ``How the}
}

\NEW{ItemFold}{\iMurkyWater}{
  \s\MYname	{A Bottle of Murky Water}
  \s\MYnumber	{1}
  \s\MYtext	{It feels cold, and tastes bitter and tinny if you take a sip. \emph{If you drink the entire bottle, open the fold and read the contents.}}
  \s\MYcontents	{The frigid water burns all the way down your throat, hitting your stomach like a weight. At first the taste is intensely bitter and metallic, but soon your mouth goes numb, and then you feel mildly dizzy. This effect will last for a few minutes. As it subsides, it is replaced by a new sensation, a vague uneasiness and sense of foreboding that you can’t quite shake.\vspace{3mm}
	
	\emph{You gain a harvest of \textcolor{yellow}{Fear}. Yellow tokens will be available at the in-game location of the Reflecting Pool for this purpose.}
	
	\vspace{3mm}
	Once you are done processing this item, {\bf keep it; the same effect will process if you drink another bottle.} }
}

\NEW{Item}{\iFlank}{
  \s\MYname	{New Action: Flank}
  \s\MYnumber	{}
  \s\MYtext	{\emph{This action is for the mechanic described in ``\gBattleStrategy{}'' and has no other in game effect.} 
	
	This action takes 2 phases to execute. In the first phase, the squad sneaks around to flank the opponent, hiding in the fog to do so. In the second phase, the squad performs an attack that is not stopped by a ``defense'' action.
	\begin{itemize}
		\item Damage dealt by the second phase of ``flank'' is determined as a normal assisted attack, and bypasses an opponents “defense” action if applicable.
		\item Flank is a damaging attack, and so it cannot be followed immediately by an ``attack'' action.
		\item {\bf If the Winter Contract is in good working order,} the squad is not target-able by {\bf damaging} actions during the {\bf first} phase of the action; any such actions targeting them will fail.
		\item This action cannot extend across rounds. Aka, you cannot use it in the 3rd phase of a round to effect the 1st phase of the next round.
	\end{itemize}
	}
}

\NEW{Item}{\iHeal}{
  \s\MYname	{New Action: Heal}
  \s\MYnumber	{}
  \s\MYtext	{\emph{This action is for the mechanic described in ``\gBattleStrategy{}'' and has no other in game effect.}
	
	This action uses Wyrd to heal squads and revive incapacitated characters. Up to 3 squads may be targeted with this action, and the targets must be chosen during the planning portion (but whether to use heal points for damage or reviving can be decided during the ``Resolving Actions'' portion.)
	\begin{itemize}
		\item Available healing points are determined as a normal assisted Wyrd check. The number of points worth of healing available is equal to the total on the check. These points are the distributed among the targeted squads at the discretion of the squad that performed the ``heal'' action.
		\item Each heal point recovers 1 HP.
		\item Reviving an incapacitated character costs 10 healing points.
		\item The squad may elect to alot zero heal points to a targeted squad in order to spend the points on a different targeted squad.
		\item Healing actions process last in a phase, allowing heal points to be applied against damage sustained in the same phase.
		\item Heal points may not be saved from phase to phase or round to round. Any unused points are lost.
	\end{itemize}
	}
}

\NEW{Item}{\iSquads}{
  \s\MYname	{New Action: 4 More Squads}
  \s\MYnumber	{}
  \s\MYtext	{\emph{This modification is for the mechanic described in ``\gBattleStrategy{}'' and has no other in game effect.}
	
	The Player Characters as a whole may now field 4 more squads, bringing the total up to 8.
	}
}

\NEW{Item}{\iBuff}{
  \s\MYname	{New Action: Buff}
  \s\MYnumber	{}
  \s\MYtext	{\emph{This action is for the mechanic described in ``\gBattleStrategy{}'' and has no other in game effect.}
	
	Multiplies by 2 the final total of {\bf another squad} that performs an assisted action in the next phase. Target must be chosen during ``planning'' portion. 
	\begin{itemize}
		\item If the targeted squad does not performing a buff-able action; nothing happens.
		\item For example, Squad A buffs Squad B in phase 1. In phase 2, Squad B performs an ``Attack'' action. They execute an assisted attack, and add $+1$ for each squad-member since the Summer Contract is in good working order. Lastly, they double their total, to determine the final amount of damage the attack does to their target. However, if Squad B defends in the second phase, then the Buff has no effect.
		\item Initially, only ``Attack'' benefits from ``Buff'', but other unlockable actions may also benefit. If you aren’t sure if ``Buff'' applies, ask a GM.
		\item {\bf If the Autumn Contract is in good working order,} buff may be used {\bf instead} to nullify the next action of a target. 
		\item This effect does not extend across rounds. Aka, you cannot use it in the 3rd phase of a round to effect the 1st phase of the next round.
	\end{itemize}
	}
}

\NEW{Item}{\iSquadResultsPatronSpecial}{
  \s\MYname	{Scout Report Three Keepers}
  \s\MYnumber	{}
  \s\MYtext	{The Patron has a special ability to double the effectiveness of its own or one its allies actions in the next phase.}
}

\NEW{Item}{\iSquadResultsStorytellerSpecial}{
  \s\MYname	{Scout Report Three Keepers}
  \s\MYnumber	{}
  \s\MYtext	{The Storyteller has a special ability that can redirect the actions of $1/3$ of the current PC squads (rounded; minimum 1) to different targets.}
}

\NEW{Item}{\iSquadResultsTrapperSpecial}{
  \s\MYname	{Scout Report Three Keepers}
  \s\MYnumber	{}
  \s\MYtext	{The Trapper has a special ability that can nullify the actions of $1/3$ of the current PC squads (rounded; minimum 1).}
}

\NEW{Item}{\iSquadResultsKeeperFlee}{
  \s\MYname	{Scout Report Three Keepers}
  \s\MYnumber	{}
  \s\MYtext	{The Keepers will flee individually when they sustain enough damage. The order in which they are driven off will affect the remainder of the battle, and you will only have a chance to kill the last remaining Keeper.}
}

\NEW{Item}{\iSquadResultsPatronBeforeTrapper}{
  \s\MYname	{Scout Report Three Keepers}
  \s\MYnumber	{}
  \s\MYtext	{If the Patron is  driven off before the Trapper, the Trapper will do more damage as the Patron is a moderating influence on the Trapper's more violent tendencies.}
}

\NEW{Item}{\iSquadResultsAttackMagnitude}{
  \s\MYname	{Scout Report Three Keepers}
  \s\MYnumber	{}
  \s\MYtext	{``Attacks'' by the Keepers will do damage \= to $1/3$ of a squad size (rounded; minimum 1 ).}
}

\NEW{Item}{\iSquadResultsSwitchTargets}{
  \s\MYname	{Scout Report Three Keepers}
  \s\MYnumber	{}
  \s\MYtext	{The same Keeper will never target the same squad twice in a row if other targets are available.}
}

\NEW{Item}{\iSquadResultsRetaliation}{
  \s\MYname	{Scout Report Three Keepers}
  \s\MYnumber	{}
  \s\MYtext	{The Storyteller and the Trapper will always use the first phase of the next round to attack a squad containing characters that dealt them damage most recently in the previous round.}
}

\NEW{Item}{\iSquadResultsStorytellerBeforePatron}{
  \s\MYname	{Scout Report Three Keepers}
  \s\MYnumber	{}
  \s\MYtext	{If the Storyteller is  driven off before the Patron, the Patron will do more damage as the Storyteller is a moderating influence on the Patron's more violent tendencies.}
}

\NEW{Item}{\iSquadResultsSpecialAbilities}{
  \s\MYname	{Scout Report Three Keepers}
  \s\MYnumber	{}
  \s\MYtext	{The Keepers will only use their special abilities once in a round, usually on the second phase.}
}

\NEW{Item}{\iSquadResultsHealthOne}{
  \s\MYname	{Scout Report Trapper and Patron}
  \s\MYnumber	{}
  \s\MYtext	{You may now use ``Scout’’ to determine remaining HP for 1 of the keepers.
	}
}

\NEW{Item}{\iSquadResultsHealthTwo}{
  \s\MYname	{Scout Report Trapper and Storyteller}
  \s\MYnumber	{}
  \s\MYtext	{You may now use ``Scout’’ to determine remaining HP for 1 of the keepers.
	}
}

\NEW{Item}{\iSquadResultsHealthThree}{
  \s\MYname	{Scout Report Patron and Storyteller}
  \s\MYnumber	{}
  \s\MYtext	{You may now use ``Scout’’ to determine remaining HP for 1 of the keepers.
	}
}

\NEW{Item}{\iSquadResultsNoStorytellerDamage}{
  \s\MYname	{Scout Report Trapper and Patron}
  \s\MYnumber	{}
  \s\MYtext	{The amount of damage the Patron does is now \= $1/2$ the squad size (rounded down; minimum 1 ).}
}

\NEW{Item}{\iSquadResultsNoStorytellerTargeting}{
  \s\MYname	{Scout Report Trapper and Patron}
  \s\MYnumber	{}
  \s\MYtext	{The keepers no longer necessarily target squads with characters that did them damage last round. Instead, in the first phase they are most likely to attack squads that have {\bf not} attempted to do them damage recently.}
}

\NEW{Item}{\iSquadResultsNoStorytellerObsession}{
  \s\MYname	{Scout Report Trapper and Patron}
  \s\MYnumber	{}
  \s\MYtext	{The keepers are likely to continue to target the same squad for multiple phases in a round.}
}

\NEW{Item}{\iSquadResultsNoStorytellerDefense}{
  \s\MYname	{Scout Report Trapper and Patron}
  \s\MYnumber	{}
  \s\MYtext	{The Trapper is less cautions and will never defend.}
}

\NEW{Item}{\iSquadResultsNoPatronDamage}{
  \s\MYname	{Scout Report Trapper and Storyteller}
  \s\MYnumber	{}
  \s\MYtext	{The amount of damage the Trapper does is now \= $1/2$ the squad size (rounded down; minimum 1 ).}
}

\NEW{Item}{\iSquadResultsNoPatronTrapperTargeting}{
  \s\MYname	{Scout Report Trapper and Storyteller}
  \s\MYnumber	{}
  \s\MYtext	{The Trapper will preferentially target squads with an even numbered designation if available.}
}

\NEW{Item}{\iSquadResultsNoPatronStorytellerTargeting}{
  \s\MYname	{Scout Report Trapper and Storyteller}
  \s\MYnumber	{}
  \s\MYtext	{The Storyteller will preferentially target squads with an odd numbered designation if available.}
}

\NEW{Item}{\iSquadResultsNoPatronDefense}{
  \s\MYname	{Scout Report Trapper and Storyteller}
  \s\MYnumber	{}
  \s\MYtext	{The Storyteller is more cautious now and will always defend in the third phase.}
}

\NEW{Item}{\iSquadResultsNoTrapperDamage}{
  \s\MYname	{Scout Report Patron and Storyteller}
  \s\MYnumber	{}
  \s\MYtext	{The amount of damage each keeper does remains $1/3$ of the squad size (rounded; minimum 1).}
}

\NEW{Item}{\iSquadResultsNoTrapperTargeting}{
  \s\MYname	{Scout Report Patron and Storyteller}
  \s\MYnumber	{}
  \s\MYtext	{The Keepers will preferentially target squads they have not targeted recently, spreading out potential damage.}
}

\NEW{Item}{\iSquadResultsNoTrapperDefense}{
  \s\MYname	{Scout Report Patron and Storyteller}
  \s\MYnumber	{}
  \s\MYtext	{The Patron is more cautious now and will always defend in the third phase.}
}

\NEW{Item}{\iSquadResultsNoTrapperSpecial}{
  \s\MYname	{Scout Report Patron and Storyteller}
  \s\MYnumber	{}
  \s\MYtext	{The Keepers can now use their special abilities twice in a round.}
}


\NEW{Item}{\iSquadResultsPatronOnlyDamage}{
  \s\MYname	{Scout Report Patron Only}
  \s\MYnumber	{}
  \s\MYtext	{The amount of damage the Patron does is now \= $1/2$ the squad size (rounded down; minimum 1 ).}
}

\NEW{Item}{\iSquadResultsPatronOnlyTargeting}{
  \s\MYname	{Scout Report Patron Only}
  \s\MYnumber	{}
  \s\MYtext	{The Patron will preferentially target squads with changeling's that used to belong to it.}
}

\NEW{Item}{\iSquadResultsTrapperOnlyDamage}{
  \s\MYname	{Scout Report Trapper Only}
  \s\MYnumber	{}
  \s\MYtext	{The amount of damage the Trapper does is now \= $1/2$ the squad size (rounded down; minimum 1 ).}
}

\NEW{Item}{\iSquadResultsSTrapperOnlyTargeting}{
  \s\MYname	{Scout Report Trapper Only}
  \s\MYnumber	{}
  \s\MYtext	{The Trapper will preferentially target squads with changeling's that used to belong to it.}
}

\NEW{Item}{\iSquadResultsStorytellerOnlyTargeting}{
  \s\MYname	{Scout Report Storyteller Only}
  \s\MYnumber	{}
  \s\MYtext	{The Storyteller will preferentially target squads with changeling's that used to belong to it.}
}

\NEW{Item}{\iSquadResultsStorytellerOnlySpecial}{
  \s\MYname	{Scout Report Storyteller Only}
  \s\MYnumber	{}
  \s\MYtext	{The Storyteller can now redirect the actions of $1/2$ of the squads, rounded down.}
}



%%%%%%%%%%%%%%%%%%%%%%%%%%%%%%%%%%%%%%%%%%%%%%%%%%%%%%%%%%%%%%%%%%