\documentclass[green]{gl2018}

\def\changemargin#1#2{\list{}{\rightmargin#2\leftmargin#1}\item[]}
\let\endchangemargin=\endlist 

\begin{document}
\name{\gBattleStrategy{}}

\section*{Introduction}
\emph{This combat system is modified from the Mouse Guard role-playing book. Acknowledgment to Jeremy Cole for help developing this mechanic.}

{\bf This is a mass combat mechanic. It is distinct from the normal combat rules and mechanic, and will be used during the climax of game. We recommend that interested parties familiarize themselves with this mechanic and take time to strategize as a group ahead of time.} If you request it, the GMs can run through a few demo rounds any time before lunch on Sunday to make sure participants understand the set up.

\section*{Structure of Combat}

Combat will be composed of a series of {\bf rounds}. Each round should last about 8 minutes and is composed of {\bf 3 portions}:
\begin{enumerate}
	\item {\bf Planning} - Each squad will have 2 minutes to pick what action the squad will perform in each of 3 phases in the second portion (see next), and decide on targets if the action requires one. Each squad must {\bf privately} chooses their actions. Members of the squad may discuss quietly among themselves, but {\bf squads may not talk to each other}. Squads may discuss potential actions for future rounds as well.
	\item {\bf Resolving Actions} - The GMs will call for first phase actions, proceeding from squad to squad, and revealing enemy actions last. 
	\begin{enumerate}
		\item Unless otherwise specified, all actions in the phase process simultaneously.
		\item All three phases will be processed sequentially.
		\item Squads cannot communicate during this portion except in direct service of resolving actions. No discussion or strategizing is allowed.
	\end{enumerate}
	\item {\bf Switching Squads} - Players will be given 2 minutes at the end of a round, to choose whether to take 1 action of: join combat (pick a squad to join), leave combat, or move to a different squad. 
	\begin{enumerate}
		\item Disbanding squads may happen as part of this round. (see below).
		\item During this phase you may share information and general strategy, but if anyone starts talking specifics for the next round of combat, the phase immediately ends.
	\end{enumerate}
\end{enumerate}

{\bf At no point may squads directly cross-talk. The only way to pass information is for someone to move from one squad to another and then share what information they choose to share.}

\section*{Organization of Player Character Resources}
\subsection*{Fielding Squads}
\begin{itemize}
	\item The players as a whole start with the ability to field up to 3 squads. 
	\item The Players must field at least 1 squad in every round, otherwise you forfeit the combat.
	\item Each squad must have {\bf at least 3 characters in it}. There is no maximum number of characters for a squad.
	\item Each squad will execute three actions each round, so fielding more squads allows you to do more actions, but each action is likely to be less effective (see below for available actions).
\end{itemize}

\subsection*{Disbanding Squads}
\begin{itemize}
	\item Squads may be disbanded by having all members choose to leave combat or join other squads.
	\item Since a squad must have a minimum of 3 characters, if two members are left after someone leaves combat or switches squads, the other two {\bf must} join a different squad or leave combat themselves.
\end{itemize}

\subsection*{Available Actions at Game Start}
Characters are assumed to be appropriately equipped for this combat with weapons and armor, and be using Contracts. The bonuses normally conveyed by these things are already incorporated into this system so they will have no additional effects. If you have ideas for items, information, or other things to impact the combat, please let the GMs know (for example, if you have the True Name of your opponent(s)) so we can give the players additional bonuses.

Squads need time to reload/regroup so {\bf squads may not perform damaging actions twice in a row within a round.}
\begin{itemize}
	\item {\bf Attack} - Does damage to a single target. Target must be chosen during the ``planning'' portion. If target is unavailable during ``resolving actions'' portion, action does nothing. 
	\begin{itemize}
		\item Damage dealt by an ``attack'' is determined as a normal assisted attack.
		\item {\bf If The Brazen Pledge is in good working order,} the total damage dealt is the assisted attack described above, $+1$ damage/ person in the squad.
		\item EXCEPTION: {\bf If The Brazen Pledge is in good working order,} the first bell chorister of journeymen or master rank in the squad adds $+2$ instead of $+1$. The second chorister adds $+3$, etc.
	\end{itemize}
	\item {\bf Defend} - Nullifies damage from the first attack received by that squad in that phase.
	\begin{itemize}
		\item {\bf If the Oath of Red Branches is in good working order,} this action will defend against all attacks that phase.
	\end{itemize}
	\item {\bf Scout} - Allows the squad to learn something about the strategy and tactics of their opponents.
	\begin{itemize}
		\item Success is determined as a successful {\bf Wyrd check of 6 or lower}. {\bf Perform this action as a single assisted action; this is the only Clarity check in game that can be assisted.} 
		\item If the squad succeeds, they will receive a piece of information, else the action accomplishes nothing.
		\item Information learned by this action is random, so it is possible to obtain redundant information, but at 2 points during the combat (will be announced), the enemy tactics will change such that old information may become irrelevant and new information will become available.
	\end{itemize}
\end{itemize}

\subsection*{Unlocking Additional Actions}
Additional action options can be unlocked by pursuing various GM and Player moderated game content. When new actions or other benefits to this mechanic are unlocked, they will be announced on the $\langle$OOC Channel$\rangle$. If you see such an announcement, find a GM and get a copy of the information for the new action or benefit.

\subsection*{Health States and Damage}
\begin{itemize}
	\item Each character involved in the combat has 3 possible health states: ``Fine'', ``Hurt'', and ``Incapacitated.'' {\bf These are independent of your normal HP.} This combat will {\bf not} affect character HP, unless the player chooses to sustain damage personally for narrative purposes.
	\item When a squad takes damage, assign the status “hurt” to a number of characters who are currently ``fine'' equal to the amount of damage sustained. If not enough ``fine'' characters are available, remaining damage is assigned to ``hurt'' players who become ``incapacitated.'' Individual characters can sustain 2 points of damage in a single round, dropping from fine'' all the way to ``incapacitated.''
	\begin{itemize}
		\item Example: \emph{A squad consists of 2 ``fine'' characters and 2 ``hurt'' characters sustains 3 damage. They must first assign the ``hurt'' status to the 2 ``fine'' characters, and then decide which of the 4 ``hurt'' characters to assign as ``incapacitated.''}
		\item ``Incapacitated'' characters don’t draw cards to support actions of a squad, but otherwise participate as normal.
		\item Incapacitated characters are still counted for the purposes of total squad count.
	\end{itemize}
	\item At game start, there is no way tor return ``hurt'' or ``incapacitated'' characters to ``fine'', but a ``heal'' action is discoverable.
	\item ``Hurt'' and ``Incapacitated'' characters may switch squads or leave combat during the appropriate portion of the round, but if they do so, they take the status with them. It cannot be healed outside of this mechanic during game, but neither does it affect their ability to participate in other parts of game. If such a player chooses to rejoin the combat at a later point, they return bearing the same status.
\end{itemize}

\end{document}
